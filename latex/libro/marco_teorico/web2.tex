%!TEX root = ../Libro.tex
\section{La Web 2.0}
El término \emph{Web 2.0} fue usado por primera vez por Dale Dougherty, vice-presidente de O'Reilly Media, en una sesión de ideas sobre el estado de la web luego de la ruptura de la ``burbuja .com'', en 2001. Dougherty notaba que la web era más importante que nunca y que las compañías que habían sobrevivido tenían muchos aspectos en común, dichas observaciones le hacían pensar que el colapso marcado por la ruptura de la burbuja determinó un punto de inflexión y que un llamado a la acción era necesario.\\

El término cobró interés desde entonces, pero hasta el momento la comunidad de desarrolladores web mantiene un desacuerdo del verdadero significado. Para zanjar de dicha diatriba, \citeauthor{Web_OReilly20052} publicó un artículo en el que aclaraba mucho de los discutido en esa sesión de ideas que dio origen al término. De dicho artículo son destacables los siete principios de la Web 2.0:

\subsection{La web como plataforma} 
\emph{Netscape Comunications}, creadores del navegador Netscape, fue la pionera en usar la web como plataforma aprovechando su dominio en el mercado de los navegadores (siendo el estandar de facto). Ofrecía un entorno (\emph{webtop}) de servicios de alto costo hospedados en el propio navegador. Al final los navegadores y los servidores se convirtieron en simple mercancía y el verdadero valor se depositó en los servicios entregados sobre la plataforma web.\\

En contraste, usando esta filosofía de entrega de servicios surgió Google. Que sin requerir instalaciones, actualizaciones o parches, ofrecía al usuario un servicio en constante mejora que generaba ganancias de manera directa o indirecta de su uso.\\

La web como plataforma se utiliza para entregar software en forma de servicios en vez de recibir servicios a través de un software.

\subsection{Aprovechando la inteligencia colectiva}
Frases como la de Eric Raymond, originalmente acuñada para referir al software de código abierto, ``con suficientes ojos, cualquier error es leve'' \citep{Neff2002} resuenan en la Web 2.0 potenciada por los usuarios: sistemas de calificación de los contenidos como el de Digg.com implementan a cabalidad la citada frase, mientras que sitios insignia de la Web 2.0 como del.icio.us y flickr.com permiten a los usuarios etiquetar los contenidos (enlaces y fotografías respectivamente) generando estructuras flexibles para la clasificación de los contenidos, denominadas folcsonomías, en contraste con las tradicionales taxonomías definidas por un administrador.

\subsection{Los datos son el próximo Intel Inside}
Todos los sitios web significativos de la actualidad se basan en el manejo de datos. Esta es la principal competencia de las iniciativas Web 2.0: el manejo de datos. La frase de Hal Varian ``SQL es el nuevo HTML'' resume la importancia que tiene el manejo de los datos.\\

Pero no sólo se trata de tener el control sobre el contenido, sino generar valor agregado a partir de dichos datos. Un ejemplo de ello es Amazon.com y Barnesandnoble.com, ambas compañías obtuvieron su catálogo de un mismo proveedor, sin embargo  Amazon se dedicó desde un comienzo a mejorar los datos agregándoles información obtenida de las casas editoriales como imágenes de las portadas, tabla de contenidos, índice y material de muestra. No sólo eso, sino que aprovecharon a sus usuarios para agregar notas a los datos de tal manera que, luego de 10 años, es Amazon y no su proveedor original quien es considerado la principal fuente para referencias bibliográficas. Adicionalmente, Amazon introdujo su propio identificador que permite referir libros que no tienen ISBN. \\ 

La carrera consiste en controlar ciertos tipos de datos: localización, identidad, fechas de eventos públicos, identificadores de productos y otros. En algunos casos el costo de crear y recopilar los datos permitirá al beneficiado mantener el control sobre ellos y licenciarlos, construyendo así lo que se denomina el Intel Inside. En los demás casos, el ganador será aquel que alcance el punto crítico por medio de sus usuarios, generando un servicio con valor agregado.

\subsection{El fin del Ciclo de Liberación del Software} 
El hecho de que el software sea entregado como servicio y no como un producto conlleva dos cambios fundamentales:

\begin{itemize}
\item \textbf{Las operaciones deben convertirse en una competencia primordial} convertir el software en un servicio requiere que diariamente sea evaluado y mantenido. Ya sea para actualizar los datos, generar nuevos derivados de ellos o simplemente mejorar la calidad de la respuesta. Esto explica que lenguajes dinámicos (también conocidos como de scripting) sean fundamentales dentro de las empresas Web 2.0 ya que dan cabida al cambio diario requerido por el software.

\item \textbf{Los usuarios deben ser tratados como co-desarrolladores} así como el paradigma del Software Libre promovía la frase ``libere temprano, libere a menudo'', el nuevo paradigma es el ``beta perpetuo'' en el que el software es desarrollado continuamente y nuevos servicios son habilitados constantemente. En este sentido, se debe mantener una vigilancia constante del patrón de uso de los servicios para facilitar la toma de decisiones sobre nuevos desarrollos.

\end{itemize}

\subsection{Modelos de desarrollo ligeros}
Desde el momento en que los servicios web se pusieron en boga, las grandes compañías se pusieron al corriente y habilitaron complejos servicios web para permitir entornos de desarrollo estables para aplicaciones distribuidas. Sin embargo, la complejidad de dichos modelos fue sustituida por opciones mas sencillas como RSS (XML).\\ 

La Web 2.0 empieza a concebir a los sistemas como consumidores de la información ofrecida libremente bajo formatos de fácil manejo, en contraste con los servicios web tradicionales que están diseñados para un alto acoplamiento. Tradicionalmente, los desarrolladores procuraban resguardar con celo sus creaciones mientras que en la actualidad, son los servicios abiertos y re-utilizables son los que se tornan más interesantes. Los servicios más exitosos son los que permiten a los usuarios mezclar los datos en maneras innovadoras.\\

La frase ``algunos derechos reservados'', popularizada por la organización \emph{Creative Commons} para contrastar con la tradicional ``todos los derechos reservados'', es la guía para la nueva era de la información abierta.

\subsection{El Software independiente del dispositivo}
La nueva plataforma no está restringida a la PC, los usuarios ahora acceden a Internet a través de teléfonos celulares, tablet PC, Blackberry y otros.\\ 

El beneficio de diseñar software independiente del dispositivo es que permitirá, no sólo el consumo desde mayor número de dispositivos sino la generación de contenidos desde más lugares: automóviles que reportan el tráfico y el periodismo ciudadano son dos ejemplos pioneros de las nuevas capacidades de la red.

\subsection{Una experiencia rica para el usuario}
Desde 1992, por medio del navegador \emph{Viola}, se había intentado ofrecer al usuario una experiencia con contenidos activos dentro del navegador usando ``applets''. Posteriormente otras opciones fueron apareciendo: Java, Javascript y Flash, las cuales permitieron al desarrollador crear aplicaciones del lado del cliente y así ofrecer experiencias más ricas para los usuarios.\\

Pero no fue hasta que Google presentó Gmail y Google Maps que estas capacidades de la plataforma fueron realmente sacadas a relucir. El uso de AJAX (Javascript asíncrono y XML) permitió crear interfaces de usuario tan ricas como las disponibles en las aplicaciones de escritorio pero con los beneficios agregados de la plataforma web: acceso consistente y desde cualquier locación.
