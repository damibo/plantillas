%!TEX root = ../Libro.tex
\subsection{Learning Objects}
Según \citeauthor{LearningOb_Beck2007}, existen 3 definiciones prominentes de qué son los learning objects:
\begin{itemize}
	\item ``Cualquier entidad, digital o no, que puede ser usado para el aprendizaje, educación o entrenamiento.''
	\item ``Cualquier recurso que puede ser reusado para dar soporte al aprendizaje''
	\item ``Una entidad digital que puede ser usada, reusada o referenciada durante el aprendizaje apoyado por la tecnología.'' 
\end{itemize}

En términos generales, sus principales características son \citep{LearningOb_Beck2007}:

\begin{itemize}
	\item \textbf{Auto-contenido y reusable}: \\ Cada learning object es independiente y puede ser compartido y reusado, en distintas ocasiones y para distintos fines, como una entidad indivisible.
	\item \textbf{Entidad de aprendizaje reducida}: \\ Al contrario de una clase completa, los learning object contienen recursos más breves de aprendizaje de unos 2 a 15 minutos.
	\item \textbf{Pueden ser agregados}: \\ Los learning objects pueden ser agrupados en colecciones más grandes, incluso en estructuras tradicionales de educación.
	\item \textbf{Rastreados con metadatos}: \\ Cada learning object contiene información descriptiva sobre la cual se pueden realizar búsquedas.
\end{itemize}