%!TEX root = ../Libro.tex
\chapter{Marco Teórico}
\section{Educación en Línea}
La educación en línea se refiere a a educación virtual, distribuída a través de Internet, la Web o cualquier otra forma de educación desarrollada con la mediación de las computadoras. Se caracteriza por:
\begin{itemize}
	\item Separación del maestro y el aprendiz, lo que la diferencia de la educación presencial.
	\item La influencia de una organización educativa, lo que la diferencia del aprendizaje autodidacta y de la tutoría privada.
	\item El uso de una red de computadores para presentar o distribuir los contenidos educativos.
	\item La disponibilidad de una vía de comunicación a través de una red de computadores de modo que los estudiantes se beneficien de la comunicación entre ellos y con el profesor.
\end{itemize}

A pesar de que el término \emph{e-learning} se usa como sinónimo de educación en línea, es importante aclarar que son distintos y que ``educación en línea'' es más amplio: \emph{e-learning} sólo se refiere a la entrega de los contenidos y la retroalimentación automatizada ante las actividades desarrolladas por el estudiante y no a la comunicación entre aprendiz y tutor. \citep{Paulsen2002}

\subsection{Learning Management Systems (LMS)}
Learning Management Systems o Sistema de Gestión del Aprendizaje, es un término amplio usado para referirse a sistemas que organizan y proveen acceso a servicios de aprendizaje en línea para estudiantes, educadores y administradores. Estos servicios incluyen control de acceso, fuente de contenidos de aprendizaje, herramientas de comunicación y organización en grupos de usuarios. También se les conoce como plataformas de aprendizaje \citep{Paulsen2002}.

\subsubsection{Objetivos}
Según \citeauthor{Hall2002}, algunos de los objetivos de un Sistema de Gestión del Aprendizaje (LMS) son:

\begin{itemize}
	\item Permitir la reutilización y redistribución de los contenidos.
	\item Ofrecer alta disponibilidad, y facilidad de uso mientras se mantiene la escalabilidad, la seguridad y la estabilidad.
	\item Integrar aprendizaje en los salones con aprendizaje en línea.
	\item Medir la efectividad de las iniciativas de aprendizaje mediante el seguimiento del progreso de los estudiantes y su desempeño en los distintos tipos de actividades. 
	\item Servir de contenedor de cursos.
	\item Facilitar la comunicación y el trabajo colaborativo entre profesores y estudiantes.
\end{itemize}
\citep{Hall2002}

\subsubsection {Características}
En cuanto a las características básicas que debe tener un LMS son destacables:

\begin{itemize}
	\item Administración de usuarios, roles, profesores, herramientas y generación de reportes.
	\item Calendario del curso.
	\item Mensajes y notificaciones a estudiantes.
	\item Autoevaluaciones y pruebas que permitan manejar el desempeño del estudiante antes y después de las evaluaciones
	\item Puntuación de actividades del curso, individual y por grupos, incluyendo lista de espera.
	\item Mostrar calificaciones y actas de notas.
	\item Distribución del curso a través de la Web o por medio de la misma.
\end{itemize}
\citep{Greenberg2002}