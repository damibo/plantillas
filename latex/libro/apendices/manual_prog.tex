%!TEX root = ../Libro.tex
\chapter{Manual del administrador}
\label{Manual}
\section{Manual de implantación}
\subsection{Requerimientos mínimos}

\subsubsection{Servidor de aplicación}
\begin{itemize}
	\item Servidor HTTP: apache 2, lighthttpd, etc.
	\item PHP 5.0 (5.2.3 recomendado) o posterior, con los módulo Rewrite (deseable) y Zip (necesario en caso de utilizar módulo ``Lecciones'').
	\item Línea de comandos de PHP (php\_cli).
	\item Acceso por shell.
	\item Subversion Client. Recomendado para la descarga de Ósmosis y CakePHP. En caso de no estar disponible, tener el \emph{tarball} de Ósmosis.
\end{itemize}

\subsubsection{Servidor de base de datos}
\begin{itemize}
	\item Cualquier servidor de base de datos relacional soportado por CakePHP: Oracle, PostgreSQL, MySQL, SQLite, etc.
\end{itemize}

\subsubsection{Cliente}
\begin{itemize}
	\item Navegador web con buen soporte para CSS2.1 y Javascript.
\end{itemize}

\subsection{Código de terceras personas}
El siguiente es un listado de librerías que utiliza y se distribuyen junto a Ósmosis2.

\subsubsection{HTMLPurifier}
\begin{itemize}
	\item Creado por: Edward Z. Yang
	\item Fecha de creación: 2006-2007
	\item Estado: en constante desarrollo.
	\item Sitio oficial: \url{http://htmlpurifier.org/}
	\item Restricciones de uso: ninguna.
	\item Licencia: GNU Lesser General Public
	\item Referencias de uso en Ósmosis: plugin wiki, plugin blog.
	\item Descripción: tiene doble función, en primer lugar corrige código HTML incorrecto para hacerlo cumplir con los estándares definidos por la W3C; y en el proceso elimina código de ataques XSS (cross-site scripting).
\end{itemize}

\subsubsection{JQuery}
\begin{itemize}
	\item Creado por: John Resig
	\item Fecha de creación: 2008
	\item Estado: en constante desarrollo.
	\item Sitio oficial: \url{http://jquery.com/}
	\item Restricciones de uso: ninguna.
	\item Licencia: dual GPL/MTI
	\item Referencias de uso en Ósmosis: visor de archivos SCORM, selector múltiple, campos autocompletados, chat.
	\item Descripción: librería JavaScript que simplifica el manejo dinámicos de los elementos del HTML, manejo de eventos, animación y Ajax.
	\item Plugins de JQuery utilizados:
	\begin{itemize}
		\item Autocomplete (\url{http://bassistance.de/jquery-plugins/jquery-plugin-autocomplete/})
		\item bgIframe (\url{http://brandonaaron.net})
		\item BlockUI (\url{http://malsup.com/jquery/block/})
		\item fieldSelection (\url{http://blog.0xab.cd})
		\item FlyDOM (\url{http://flydom.socianet.com/})		
		\item Metadata (\url{http://plugins.jquery.com/project/metadata})
		\item OsmosisSelector
		\item Star Rating (\url{http://www.fyneworks.com/jquery/star-rating/})
		\item ScrollTo (\url{http://flesler.blogspot.com/2007/10/jqueryscrollto.html})
	\end{itemize}
\end{itemize}

\subsubsection{TinyMCE}
\begin{itemize}
	\item Creado por: Moxiecode Systems AB
	\item Fecha de creación: no registrada
	\item Estado: en constante desarrollo.
	\item Sitio oficial: \url{http://tinymce.moxiecode.com/}
	\item Restricciones de uso: ninguna.
	\item Licencia: GNU Lesser General Public
	\item Referencias de uso en Ósmosis: plugin wiki.
	\item Descripción: editor Javascript WYSIWYG, basado en la web, para HTML. Tiene la capacidad de convertir los textarea en editores de texto enriquecido.
\end{itemize}

\subsection{Estructura del sistema}
Ósmosis2 mantiene la estructura básica de una aplicación desarrollada con CakePHP \citep{CakePHP_Folders_2008}:
\begin{itemize}
	\item app  (se referirá a este directorio como APP)
	\item cake (se referirá a este directorio como CAKE\_DIR)
	\item docs
	\item index.php
	\item vendors
\end{itemize}

Sin embargo, Ósmosis2 no se distribuye junto con las librerías que conforman el core de CakePHP. Sólo se distribuye el equivalente al contenido del directorio \emph{app}.\\

\subsection{Instalación}
El proceso de instalación de Ósmosis se puede dividir en cuatro secciones:
\begin{enumerate}
	\item Instalación de CakePHP
	\item Instalación de Ósmosis
	\item Configuración de Ósmosis
	\item Instalación y habilitación de Plugins
\end{enumerate}

\subsubsection{Instalación de CakePHP}
Existen múltiples maneras de instalar CakePHP \citep{CakePHP_Install_2008}, el definir cuál es la más adecuada dependerá de las capacidades administrativas que se tengan sobre el servidor.

\subsubsection{Instalación de Ósmosis}
Una vez verificada la correcta instalación de CakePHP, es necesario sustituir el directorio \emph{app} que se distribuye con Cake por Ósmosis.\\

\begin{lstlisting}
rm -rf app
svn co http://tools.assembla.com/svn/osmosis/trunk/osmosis app
\end{lstlisting}

En caso de no poseer el cliente de subversion:

\begin{lstlisting}
rm -rf app
tar -xzvf osmosis.tar.gz
\end{lstlisting}

Con cualquiera de las dos opciones el directorio app deberá contener los archivos de Ósmosis. También se debe asegurar que la carpeta tmp dentro del directorio de Ósmosis tenga permisos de escritura por el usuario que corre el servidor. En linux esto podría ser como sigue:

\begin{lstlisting}
chown -R www-data app/tmp
\end{lstlisting}

\subsubsection{Configuración de Ósmosis}
La configuración inicial de Ósmosis consiste en la modificación del \textbf{archivo APP/config/database.php.default}, la cual tiene instrucciones claras para la configuración.\\

Una vez modificado este archivo, debe guardarse como \textbf{database.php}
\begin{lstlisting}[language=PHP]
class DATABASE_CONFIG {
	var $default = array(
		'driver'	 => 'mysql',
		// Alguno de los drivers soportados: mysql, postgres, oracle ...
		'persistent' => false,
		// Utilizar o no conexiones persistentes
		'host'		 => 'localhost',
		// Host del servidor de bases de datos
		'login'		 => 'user',
		// Nombre de usuario de la base de datos
		'password'	 => 'password',
		// Clave del usuario de la base de datos
		'database'	 => 'name',
		// Nombre de la base de datos
		'prefix'	 => ''
		// Prefijo de las tablas (opcional)
	);
}
\end{lstlisting}

Una vez configurada la conexión a la base de datos, es necesario cargar las tablas. Para ello Ósmosis hace uso de los esquemas de base de datos de CakePHP.\\

\begin{lstlisting}
alias cake=CAKE_DIR/console/cake
cake schema run create
\end{lstlisting}

Siguiendo las instrucciones disponibles en la consola se cargarán las tablas en la base de datos configurada.

El último paso del proceso de configuración de Ósmosis consiste en ejecutar la acción init del controlador InitAcl, la cual configura la base de datos de permisos y crea el primer usuario administrador. Para ejecutar dicha acción apunte el navegador a \texttt{www.example.com/initAcl/init}.\\

Para poder ejecutar dicha acción es necesario modificar temporalmente el archivo \textbf{APP/config/core.php} para deshabilitar la autenticación:
\begin{lstlisting}[language=PHP]
	Configure::write('Auth.disabled', true);
\end{lstlisting}

No olvide reactivar la autenticación posteriormente:
\begin{lstlisting}[language=PHP]
	Configure::write('Auth.disabled', false);
\end{lstlisting}

Por razones de seguridad se recomienda cambiar el valor del \emph{sal} de seguridad en el archivo \textbf{APP/config/core.php} por cualquier cadena de caracteres suficientemente aleatoria:
\begin{lstlisting}[language=PHP]
	Configure::write('Security.salt', '$hl56MMsdb0qyJfIxfs2guVoUubWwvniR2G0FgaC9mi');
\end{lstlisting}

Esta cadena de caracteres es usada como semilla de encriptamiento de las claves de usuarios, y para evitar robo de sesiones, entre otras, por lo que si se deja igual a como es distribuido actualmente el paquete de Ósmosis se deja más vulnerable la aplicación.

\subsubsection{Instalación y habilitación de plugins}
Las herramientas que posee Ósmosis2 están construidas utilizando su arquitectura de plugins. La lista de plugins distribuidos oficialmente es la siguiente:
\begin{itemize}
	\item Agenda
	\item Blog
	\item Chat
	\item Foro
	\item Locker
	\item Quiz
	\item Scorm
	\item Wiki
\end{itemize}

Si desea instalar otro plugin debe colocarlo en el directorio \textbf{APP/plugins} junto con los anteriores. Una vez colocados los plugins correctamente, estarán disponibles, para ser instalados, en el área administrativa de plugins (www.example.com/admin/plugins). \\

\section{Desarrollo de Plugins}

\subsubsection{Estructura Inicial}
CakePHP ofrece a los desarrolladores una herramienta de línea de comandos, la cónsola de Cake, que permite agilizar el proceso de desarrollo de aplicaciones. Una de las facilidades que ofrece es la generación de la estructura de archivos de un plugin de CakePHP.

\begin{lstlisting}
cake bake plugin nombre_plugin
\end{lstlisting}

Esto genera un directorio ``nombre\_plugin'' en el directorio plugins de Ósmosis.

\subsubsection{Descripción del Plugin}
Una vez creada la estructura básica del plugin, es recomendable crear el archivo \textbf{APP/plugins/nombre\_plugin/config/description.php} en el cual se colocaran los siguientes descriptores del plugin.\\

\begin{lstlisting}[language=PHP]
<?php
Configure::write(
	'NombrePlugin.description',
	'Lorem ipsum dolor sit amet, consectetuer adipiscing elit.'
);
Configure::write(
	'NombrePlugin.title',
	'Nombre del Plugin'
);
Configure::write(
	'NombrePlugin.type',
	array('tool')
);
Configure::write(
	'NombrePlugin.author',
	'Osmosis Team'
);
?>
\end{lstlisting}

Todos los valores son de texto libre, excepto \emph type que debe ser un arreglo. El los valores aceptados \emph{type} son `tool' si desea que aparezca en la barra de herramientas, sino marcarlo como `other'.\\

La existencia de este archivo es recomendable, en caso de no existir se tratará el plugin como `other' y no se mostrará la descripción.

\subsubsection{Estructura de la Base de Datos}
Ósmosis no impone restricciones explícitas sobre la estructura de la base de datos de un plugin, sin embargo se recomienda:
\begin{itemize}
	\item Seguir las convenciones de CakePHP.
	\item Prefijar los nombres de las tablas con el nombre del plugin, de este modo se facilita la identificación de las tablas.
\end{itemize}

Para la carga de las tablas de los plugins se recomienda usar los schemas de base de datos de CakePHP, ya que son agnósticos al manejador de base de datos y permiten automatizaciones interesantes.\\

La generación de los esquemas de base de datos se hacen con la cónsola de Cake, desde el directorio en el cual se quiere crear el schema.

\begin{lstlisting}
cake schema generate
\end{lstlisting}

El schema creado debe ubicarse en el directorio \emph{/config/sql} del plugin.

\section[Internacionalización]{Internacionalización de la plataforma}
Los mensajes visibles por el usuario en Ósmosis están totalidad en inglés, y se recomienda seguir esta convención para los plugins. Sin embargo, CakePHP permite la internacionalización de sus aplicaciones por medio de archivos de traducción.

\subsection{Funciones de Internacionalización}
Dentro de cualquier sección de una aplicación Cake está disponible un juego de funciones para la internacionalización de las cadenas de caracteres. De dichas funciones, la más utilizada es \_\_() la cual recibe dos parámetros:
\begin{enumerate}
	\item Cadena de caracteres internacionalizable
	\item Booleano que determina si la función debe imprimir o devolver la cadena traducida
\end{enumerate}

Un ejemplo del uso de esta función:
\begin{lstlisting}[language=PHP]
<?php
	$del = __('Delete', true); //La variable $del contiene la palabra traducida
	__('Delete'); // Se imprime la palabra traducida
?>
\end{lstlisting}

\subsection{Extracción de cadenas internacionalizadas}
CakePHP ofrece otra herramienta para ayudar a los desarrolladores a generar los archivos para traducir sus aplicaciones. Nuevamente, utilizado la consola de CakePHP.

\begin{lstlisting}
cake i18n
\end{lstlisting}

Siguiendo las instrucciones se generará el archivo de traducciones.
Cada plugin tiene un archivo de traducción, por lo cual, al generarlo, se debe indicar la ruta completa hasta el plugin que se desea traducir. Por ejemplo /Users/user/Sites/osmosis/agenda, en donde agenda es el plugin seleccionado. Así mismo, para llevar a cabo la traducción se debe contar con la misma estructura de archivos que posee la aplicación principal.
El archivo de traducción, \textbf{default.pot}, se generará en la carpeta \textbf{locale} de cada plugin.


\subsection{Funcionamiento de los archivos de traducción de cake}
Los archivos para manejar la traducción del lenguaje se encuentran en el directorio \textbf{locale}. Cada lenguaje es un subdirectorio dentro de dicha carpeta, cuyo nombre consta de tres caracteres correspondiente al al nombre del idioma al cual se hace referencia, siguiendo el estándar para los códigos de representación de nombres de idiomas (ISO 639-2)\citep{MAN_INT}. Por ejemplo \emph{eng} para el idioma inglés (english), \emph{ita} para el idioma italiano (italian)

En el subdirectorio del idioma se debe crear una directorio de nombre \textbf{LC\_MESSAGES} que contendrá el archivo ``default.po''. Este archivo consta de cadenas de caracteres con claves. 

Los archivos .po deben tener un encoding ISO-8859-1 y los strings debe ser menores a 1014 caracteres, a continuación se muestra un ejemplo de cómo se ven las cadenas de caracteres y las claves de las mismas en el archivo .po: 
\begin{lstlisting}{}
msgid "Hello World"
msgstr "Hola Mundo"
\end{lstlisting}

El archivo de traducción que se obtiene a partir de esta función toma el primer parámetro como el la clave a la cual se le asignará el mensaje traducido correspondiente.

\subsection{Cómo traducir}
Para generar la traduccion completa de la plataforma, es necesario completar cada par msgid - msgstr, donde el msgid siempre queda fijo por ser la clave usada dentro de la plataforma, y msgstr el mensaje correspondiente a la traducción que se desea construir. En algunos casos es posible encontrar dentro de la cadena msgid subcadenas de reemplazo, que son palabras que comienzan por \%, usualmente siempre seguidas de una única letra \textbf{s}. Estas cadenas de reemplazo representan una o varias palabras, números, o cualquier otra cosa que haga sentid dentro de la construcción. Un ejemplo de cadena de reemplazo es el siguiente\\

\begin{lstlisting}{}
msgid "Hello %s"
msgstr "Hola %s"
\end{lstlisting}

Donde la cadena de reemplazo \%s, puede representar el nombre de una persona.\\

Esto existe para la facilidad del programador, pues delega el trabajo de traducción a varios pares clave-valor de traduccion, no obstante recomendamos hacer el menor uso posible de estas construcciones, puesto que aunque es posible que para el idioma en que se haya construido el msgid tenga sentido ese reemplazo, para otros idiomas con gramáticas distintas puede ser muy engorroso saber la estructura gramatical adecuada que se debe usar.\\

\subsection{Cómo generar el archivo binario (.mo)}
El archivo con extensión \textbf{.mo} se genera en el mismo directorio que contiene al archivo de extensión .po. Para generarlo se emplea el comando 
\begin{lstlisting}{}
'msgfmt -o target.mo source.po'
\end{lstlisting}

\section{Más información}
CakePHP tiene a disposición de los desarrolladores el manual del framework en la dirección \url{http://book.cakephp.org/}. En este manual se explica todo lo relacionado a CakePHP, su instalación, configuración y convenciones así como el desarrollo de aplicaciones bajo este framework.

\section{Diccionario de datos}
\subsection{Tablas de ACL}
% 
% Database: 'osmosis'
%
% Structure: acos
%
\begin{longtable}{c c c c l}
	\multicolumn{1}{c}{\textbf{Field}} &
	\multicolumn{1}{c}{\textbf{Type}} &
	\multicolumn{1}{c}{\textbf{Null}} &
	\multicolumn{1}{c}{\textbf{Default}} &
	\multicolumn{1}{c}{\textbf{Comments}} \\ \hline \hline
\endfirsthead
	\multicolumn{1}{c}{\textbf{Field}} &
	\multicolumn{1}{c}{\textbf{Type}} &
	\multicolumn{1}{c}{\textbf{Null}} &
	\multicolumn{1}{c}{\textbf{Default}} &
	\multicolumn{1}{c}{\textbf{Comments}} \\ \hline \hline
\endhead \endfoot
	\textbf{\textit{id}} & int(11) & Yes & NULL \\ \hline 
	parent\_id & int(11) & Yes & NULL \\ \hline 
	model & varchar(255) & Yes &  \\ \hline 
	foreign\_key & int(11) & Yes & NULL \\ \hline 
	alias & varchar(255) & Yes &  \\ \hline 
	lft & int(11) & Yes & NULL \\ \hline 
	rght & int(11) & Yes & NULL \\ \\ 
\caption[Estructura de la tabla acos]{Estructura de la tabla acos. Los Access Control Objects son los objetos cuyo acceso debe restringirse \citep{CakePHP_ACL_2008}} \label{tab:acos-structure} \\
\end{longtable}

%
% Structure: aros
%
\begin{longtable}{c c c c l}
	\multicolumn{1}{c}{\textbf{Field}} &
	\multicolumn{1}{c}{\textbf{Type}} &
	\multicolumn{1}{c}{\textbf{Null}} &
	\multicolumn{1}{c}{\textbf{Default}} &
	\multicolumn{1}{c}{\textbf{Comments}} \\ \hline \hline
\endfirsthead
	\multicolumn{1}{c}{\textbf{Field}} &
	\multicolumn{1}{c}{\textbf{Type}} &
	\multicolumn{1}{c}{\textbf{Null}} &
	\multicolumn{1}{c}{\textbf{Default}} &
	\multicolumn{1}{c}{\textbf{Comments}} \\ \hline \hline
\endhead \endfoot
	\textbf{\textit{id}} & int(11) & Yes & NULL \\ \hline 
	parent\_id & int(11) & Yes & NULL \\ \hline 
	model & varchar(255) & Yes &  \\ \hline 
	foreign\_key & int(11) & Yes & NULL \\ \hline 
	alias & varchar(255) & Yes &  \\ \hline 
	lft & int(11) & Yes & NULL \\ \hline 
	rght & int(11) & Yes & NULL \\ \\ 
\caption[Estructura de la tabla aros]{Estructura de la tabla aros. Los Access Request Objects son los entes que requieren acceso a los objetos restringidos \citep{CakePHP_ACL_2008}.} \label{tab:aros-structure} \\ 
\end{longtable}

%
% Structure: aros_acos
%
\begin{longtable}{c c c c l}
	\multicolumn{1}{c}{\textbf{Field}} &
	\multicolumn{1}{c}{\textbf{Type}} &
	\multicolumn{1}{c}{\textbf{Null}} &
	\multicolumn{1}{c}{\textbf{Default}} &
	\multicolumn{1}{c}{\textbf{Comments}} \\ \hline \hline
\endfirsthead
	\multicolumn{1}{c}{\textbf{Field}} &
	\multicolumn{1}{c}{\textbf{Type}} &
	\multicolumn{1}{c}{\textbf{Null}} &
	\multicolumn{1}{c}{\textbf{Default}} &
	\multicolumn{1}{c}{\textbf{Comments}} \\ \hline \hline
\endhead \endfoot
	\textbf{\textit{id}} & int(11) & Yes & NULL \\ \hline
	aro\_id & int(11) & Yes & NULL \\ \hline 
	aco\_id & int(11) & Yes & NULL \\ \hline 
	\_create & int(11) & Yes & 0 \\ \hline 
	\_read & int(11) & Yes & 0 \\ \hline 
	\_update & int(11) & Yes & 0 \\ \hline 
	\_delete & int(11) & Yes & 0 \\ \\ 
\caption[Estructura de la tabla aros\_acos]{Estructura de la tabla aros\_acos. Establece los permisos de un ARO sobre cada ACO.} \label{tab:aros_acos-structure} \\ 
\end{longtable}


\subsection{Tablas del Núcleo de Ósmosis}
%
% Structure: members
%
\begin{longtable}{c c c c l}
	\multicolumn{1}{c}{\textbf{Field}} &
	\multicolumn{1}{c}{\textbf{Type}} &
	\multicolumn{1}{c}{\textbf{Null}} &
	\multicolumn{1}{c}{\textbf{Default}} &
	\multicolumn{1}{c}{\textbf{Comments}} \\ \hline \hline
\endfirsthead
	\multicolumn{1}{c}{\textbf{Field}} &
	\multicolumn{1}{c}{\textbf{Type}} &
	\multicolumn{1}{c}{\textbf{Null}} &
	\multicolumn{1}{c}{\textbf{Default}} &
	\multicolumn{1}{c}{\textbf{Comments}} \\ \hline \hline
\endhead \endfoot
	\textbf{\textit{id}} & int(11)  & Yes & NULL & \parbox[t]{0.35\textwidth}{Identificador del miembro} \\ \\ \hline
	institution\_id & varchar(20) & Yes & NULL & \parbox[t]{0.35\textwidth}{Identificador del miembro dentro de la institución (canré)} \\ \\ \hline
	full\_name & varchar(50) & Yes & NULL & \parbox[t]{0.35\textwidth}{Nombre completo del miembro} \\ \\ \hline
	email & varchar(50) & Yes & NULL & \parbox[t]{0.35\textwidth}{Dirección de e-mail del miembro} \\ \\ \hline
	phone & varchar(20) & Yes & NULL & \parbox[t]{0.35\textwidth}{Número telefónico del miembro} \\ \\ \hline
	country & varchar(20) & Yes & NULL & \parbox[t]{0.35\textwidth}{País de nacimiento} \\ \\  \hline
	city & varchar(50) & Yes & NULL & \parbox[t]{0.35\textwidth}{Ciudad de nacimiento} \\ \\ \hline 
	age & int(2) & Yes & NULL & \parbox[t]{0.35\textwidth}{Edad} \\ \\  \hline
	sex & varchar(1) & Yes & M & \parbox[t]{0.35\textwidth}{Sexo (M o F)} \\ \\  \hline
	username & varchar(15) & Yes & NULL & \parbox[t]{0.35\textwidth}{Nombre de usuario (Login)} \\ \\  \hline
	password & varchar(50) & Yes & NULL & \parbox[t]{0.35\textwidth}{Contraseña} \\ \\
\caption[Estructura de la tabla members]{Estructura de la tabla members. Mantiene los datos de los miembros registrados en Ósmosis} \label{tab:members-structure} \\  
\end{longtable}

%
% Structure: plugins
%
\begin{longtable}{c c c c l}
	\multicolumn{1}{c}{\textbf{Field}} &
	\multicolumn{1}{c}{\textbf{Type}} &
	\multicolumn{1}{c}{\textbf{Null}} &
	\multicolumn{1}{c}{\textbf{Default}} &
	\multicolumn{1}{c}{\textbf{Comments}} \\ \hline \hline
\endfirsthead
	\multicolumn{1}{c}{\textbf{Field}} &
	\multicolumn{1}{c}{\textbf{Type}} &
	\multicolumn{1}{c}{\textbf{Null}} &
	\multicolumn{1}{c}{\textbf{Default}} &
	\multicolumn{1}{c}{\textbf{Comments}} \\ \hline \hline
\endhead \endfoot
	\textbf{\textit{id}} & smallint(4)  & Yes & NULL & \parbox[t]{0.35\textwidth}{Identificador del plugin} \\ \\  \hline
	title & varchar(50) & Yes & NULL & \parbox[t]{0.35\textwidth}{Título del plugin} \\ \\  \hline
	active & tinyint(1) & Yes & NULL & \parbox[t]{0.35\textwidth}{Activación global del plugin} \\ \hline 
	name & varchar(100) & Yes & NULL & \parbox[t]{0.35\textwidth}{Nombre del plugin} \\ \\  \hline
	description & varchar(255) & Yes & NULL & \parbox[t]{0.35\textwidth}{Descripción del plugin} \\ \\  \hline
	author & varchar(100) & Yes & NULL & \parbox[t]{0.35\textwidth}{Autor} \\ \hline 
	types & varchar(30) & Yes & tool & \parbox[t]{0.35\textwidth}{Tipo del plugin} \\ \\  \hline
\caption[Estructura de la tabla plugins]{Estructura de la tabla plugins. Maneja los plugins habilitados para la plataforma.} \label{tab:plugins-structure} \\
\end{longtable}

%
% Structure: courses
%
\begin{longtable}{c c c c l}
	\multicolumn{1}{c}{\textbf{Field}} &
	\multicolumn{1}{c}{\textbf{Type}} &
	\multicolumn{1}{c}{\textbf{Null}} &
	\multicolumn{1}{c}{\textbf{Default}} &
	\multicolumn{1}{c}{\textbf{Comments}} \\ \hline \hline
\endfirsthead
	\multicolumn{1}{c}{\textbf{Field}} &
	\multicolumn{1}{c}{\textbf{Type}} &
	\multicolumn{1}{c}{\textbf{Null}} &
	\multicolumn{1}{c}{\textbf{Default}} &
	\multicolumn{1}{c}{\textbf{Comments}} \\ \hline \hline
\endhead \endfoot
	\textbf{\textit{id}} & int(11)  & Yes & NULL & \parbox[t]{0.35\textwidth}{Identificador del curso} \\ \\  \hline
	department\_id & int(4)  & Yes & NULL & \parbox[t]{0.35\textwidth}{Identificador del departamento al cual esta asociado el curso} \\ \\  \hline
	owner\_id & int(11) & Yes & NULL & \parbox[t]{0.35\textwidth}{Identificador del miembro creador del curso} \\ \\  \hline
	code & varchar(10) & Yes & NULL & \parbox[t]{0.35\textwidth}{Código del curso} \\ \\  \hline
	name & varchar(100) & Yes & NULL & \parbox[t]{0.35\textwidth}{Nombre del curso} \\ \\  \hline
	description & text & Yes & NULL & \parbox[t]{0.35\textwidth}{Descripción del curso} \\ \\  \hline
	created & date & Yes & NULL & \parbox[t]{0.35\textwidth}{Fecha de creación del curso} \\ \\  \hline
 \caption[Estructura de la tabla courses]{Estructura de la tabla courses. Cursos registrados en la plataforma} \label{tab:courses-structure} \\
\end{longtable}

%
% Structure: courses_members
%
\begin{longtable}{c c c c l}
	\multicolumn{1}{c}{\textbf{Field}} &
	\multicolumn{1}{c}{\textbf{Type}} &
	\multicolumn{1}{c}{\textbf{Null}} &
	\multicolumn{1}{c}{\textbf{Default}} &
	\multicolumn{1}{c}{\textbf{Comments}} \\ \hline \hline
\endfirsthead
	\multicolumn{1}{c}{\textbf{Field}} &
	\multicolumn{1}{c}{\textbf{Type}} &
	\multicolumn{1}{c}{\textbf{Null}} &
	\multicolumn{1}{c}{\textbf{Default}} &
	\multicolumn{1}{c}{\textbf{Comments}} \\ \hline \hline
\endhead \endfoot
	\textbf{\textit{id}} & char(36) & Yes & NULL & \parbox[t]{0.35\textwidth}{Identificador de la relación entre los cursos y los miembros}\\ \hline 
	member\_id & int(11) & Yes & NULL & \parbox[t]{0.35\textwidth}{Identificador del miembro} \\ \\  \hline
	course\_id & int(11) & Yes & NULL & \parbox[t]{0.35\textwidth}{Identificador del curso} \\ \\  \hline 
	role\_id & int(11) & Yes & NULL & \parbox[t]{0.35\textwidth}{Identificador del rol del miembro en ese curso} \\ \\ 
\caption[Estructura de la tabla members]{Estructura de la tabla courses\_members. Mantiene la matrícula (Enrollments) de miembros en los cursos de la plataforma} \label{tab:courses_members-structure} \\
\end{longtable}

%
% Structure: course_tools
%
\begin{longtable}{c c c c l}
	\multicolumn{1}{c}{\textbf{Field}} &
	\multicolumn{1}{c}{\textbf{Type}} &
	\multicolumn{1}{c}{\textbf{Null}} &
	\multicolumn{1}{c}{\textbf{Default}} &
	\multicolumn{1}{c}{\textbf{Comments}} \\ \hline \hline
\endfirsthead
	\multicolumn{1}{c}{\textbf{Field}} &
	\multicolumn{1}{c}{\textbf{Type}} &
	\multicolumn{1}{c}{\textbf{Null}} &
	\multicolumn{1}{c}{\textbf{Default}} &
	\multicolumn{1}{c}{\textbf{Comments}} \\ \hline \hline
\endhead \endfoot
	\textbf{\textit{id}} & int(10) & Yes & NULL & \parbox[t]{0.35\textwidth}{Identificador de la relación entre los cursos y las herramientas} \\ \hline 
	\textbf{course\_id} & int(11) & Yes & NULL & \parbox[t]{0.35\textwidth}{Identificador del curso} \\ \\  \hline
	\textbf{plugin\_id} & smallint(4) & Yes & NULL & \parbox[t]{0.35\textwidth}{Identificador del plugin} \\ \\ 
\caption[Estructura de la tabla course\_tools]{Estructura de la tabla course\_tools. Maneja los plugins (tipo tool) activados por curso.} \label{tab:course_tools-structure} \\
\end{longtable}

%
% Structure: departments
%
\begin{longtable}{c c c c l}
	\multicolumn{1}{c}{\textbf{Field}} &
	\multicolumn{1}{c}{\textbf{Type}} &
	\multicolumn{1}{c}{\textbf{Null}} &
	\multicolumn{1}{c}{\textbf{Default}} &
	\multicolumn{1}{c}{\textbf{Comments}} \\ \hline \hline
\endfirsthead
	\multicolumn{1}{c}{\textbf{Field}} &
	\multicolumn{1}{c}{\textbf{Type}} &
	\multicolumn{1}{c}{\textbf{Null}} &
	\multicolumn{1}{c}{\textbf{Default}} &
	\multicolumn{1}{c}{\textbf{Comments}} \\ \hline \hline
\endhead \endfoot
	\textbf{\textit{id}} & int(4) & Yes & NULL & \parbox[t]{0.35\textwidth}{Identificador del departamento} \\ \\  \hline
	name & varchar(150) & Yes & NULL & \parbox[t]{0.35\textwidth}{Nombre del departamento} \\ \\  \hline
	description & text & Yes & NULL & \parbox[t]{0.35\textwidth}{Descripción del departamento} \\ \\  \hline
\caption[Estructura de la tabla departments]{Estructura de la tabla departments. Maneja los departamentos que agrupan los cursos.} \label{tab:departments-structure} \\ 
\end{longtable}

%
% Structure: roles
%
\begin{longtable}{c c c c l}
	\multicolumn{1}{c}{\textbf{Field}} &
	\multicolumn{1}{c}{\textbf{Type}} &
	\multicolumn{1}{c}{\textbf{Null}} &
	\multicolumn{1}{c}{\textbf{Default}} &
	\multicolumn{1}{c}{\textbf{Comments}} \\ \hline \hline
\endfirsthead
	\multicolumn{1}{c}{\textbf{Field}} &
	\multicolumn{1}{c}{\textbf{Type}} &
	\multicolumn{1}{c}{\textbf{Null}} &
	\multicolumn{1}{c}{\textbf{Default}} &
	\multicolumn{1}{c}{\textbf{Comments}} \\ \hline \hline
\endhead \endfoot
	\textbf{\textit{id}} & int(11)  & Yes & NULL & \parbox[t]{0.35\textwidth}{Identificador del Rol} \\ \\ \hline 
	parent\_id & int(11) & Yes & NULL & \parbox[t]{0.35\textwidth}{Identificador del Rol padre} \\ \\ \hline 
	role & varchar(10) & Yes & NULL & \parbox[t]{0.35\textwidth}{Nombre del rol} \\ \\ 
 \caption[Estructura de la tabla roles]{Estructura de la tabla roles. Mantiene una estructura arborescente de los roles del sistema} \label{tab:roles-structure} \\
\end{longtable}

\subsection{Tablas relacionadas al plugin Blog}
%
% Structure: blog_blogs
%
\begin{longtable}{c c c c l}
	\multicolumn{1}{c}{\textbf{Field}} &
	\multicolumn{1}{c}{\textbf{Type}} &
	\multicolumn{1}{c}{\textbf{Null}} &
	\multicolumn{1}{c}{\textbf{Default}} &
	\multicolumn{1}{c}{\textbf{Comments}} \\ \hline \hline
\endfirsthead
	\multicolumn{1}{c}{\textbf{Field}} &
	\multicolumn{1}{c}{\textbf{Type}} &
	\multicolumn{1}{c}{\textbf{Null}} &
	\multicolumn{1}{c}{\textbf{Default}} &
	\multicolumn{1}{c}{\textbf{Comments}} \\ \hline \hline
\endhead \endfoot
	\textbf{\textit{id}} & int(11) & Yes & NULL & \parbox[t]{0.35\textwidth}{Identificador del blog} \\ \\  \hline
	title & varchar(200) & Yes & NULL & \parbox[t]{0.35\textwidth}{Título del blog} \\ \\  \hline
	description & text & Yes & NULL & \parbox[t]{0.35\textwidth}{Descripción del blog} \\ \\  \hline
	member\_id & varchar(100) & Yes & NULL & \parbox[t]{0.35\textwidth}{Identificador del miembro propietario del blog} \\ \\
\caption[Estructura de la tabla blogs]{Estructura de la tabla blogs. Mantiene el listado de los blogs de los miembros.} \label{tab:blog_blogs-structure} \\
\end{longtable}

%
% Structure: blog_comments
%
\begin{longtable}{c c c c l}
	\multicolumn{1}{c}{\textbf{Field}} &
	\multicolumn{1}{c}{\textbf{Type}} &
	\multicolumn{1}{c}{\textbf{Null}} &
	\multicolumn{1}{c}{\textbf{Default}} &
	\multicolumn{1}{c}{\textbf{Comments}} \\ \hline \hline
\endfirsthead
	\multicolumn{1}{c}{\textbf{Field}} &
	\multicolumn{1}{c}{\textbf{Type}} &
	\multicolumn{1}{c}{\textbf{Null}} &
	\multicolumn{1}{c}{\textbf{Default}} &
	\multicolumn{1}{c}{\textbf{Comments}} \\ \hline \hline
\endhead \endfoot
	\textbf{\textit{id}} & int(11) & Yes & NULL & \parbox[t]{0.35\textwidth}{Identificador del comentario} \\ \\  \hline
	comment & text & Yes & NULL & \parbox[t]{0.35\textwidth}{Identificador del comentario} \\ \\  \hline
	post\_id & int(11) & Yes & NULL & \parbox[t]{0.35\textwidth}{Identificador de la entrada al la cual está asociado el comentario} \\ \\  \hline
	member\_id & int(11) & Yes & NULL & \parbox[t]{0.35\textwidth}{Identificador del miembro que realizó el comentario} \\ \\
\caption[Estructura de la tabla comments]{Estructura de la tabla comments. Almacena los comentarios a las entradas} \label{tab:blog_comments-structure} \\
\end{longtable}

%
% Structure: blog_posts
%
\begin{longtable}{c c c c l}
	\multicolumn{1}{c}{\textbf{Field}} &
	\multicolumn{1}{c}{\textbf{Type}} &
	\multicolumn{1}{c}{\textbf{Null}} &
	\multicolumn{1}{c}{\textbf{Default}} &
	\multicolumn{1}{c}{\textbf{Comments}} \\ \hline \hline
\endfirsthead
	\multicolumn{1}{c}{\textbf{Field}} &
	\multicolumn{1}{c}{\textbf{Type}} &
	\multicolumn{1}{c}{\textbf{Null}} &
	\multicolumn{1}{c}{\textbf{Default}} &
	\multicolumn{1}{c}{\textbf{Comments}} \\ \hline \hline
\endhead \endfoot
	\textbf{\textit{id}} & int(10)  & Yes & NULL & \parbox[t]{0.35\textwidth}{Identificador de la entrada} \\ \\  \hline
	title & varchar(50) & Yes & NULL & \parbox[t]{0.35\textwidth}{Título de la entrada} \\ \\  \hline
	body & text & Yes & NULL & \parbox[t]{0.35\textwidth}{Cuerpo o contenido de la entrada} \\ \\  \hline
	created & datetime & Yes & NULL & \parbox[t]{0.35\textwidth}{Fecha de creación de la entrada} \\ \\  \hline
	modified & datetime & Yes & NULL & \parbox[t]{0.35\textwidth}{Fecha de modificación de la entrada} \\ \\  \hline
	blog\_id & int(11) & Yes & NULL & \parbox[t]{0.35\textwidth}{Identificador del blog al cual pertenece la entrada} \\ \\  \hline
	slug & text & Yes & NULL & \parbox[t]{0.35\textwidth}{Versión para URLs del título de la entrada} \\ \\  \hline
	member\_id & int(11) & Yes & NULL & \parbox[t]{0.35\textwidth}{Identificador del usuario que escribió la entrada} \\ \\
\caption[Estructura de la tabla posts]{Estructura de la tabla posts. Almacena las entradas de todos los blogs} \label{tab:blog_posts-structure} \\
\end{longtable}

\subsection{Tablas relacionadas al plugin Foro}
%
%
% Structure: forum_discussions
%
\begin{longtable}{c c c c l}
	\multicolumn{1}{c}{\textbf{Field}} &
	\multicolumn{1}{c}{\textbf{Type}} &
	\multicolumn{1}{c}{\textbf{Null}} &
	\multicolumn{1}{c}{\textbf{Default}} &
	\multicolumn{1}{c}{\textbf{Comments}} \\ \hline \hline
\endfirsthead
	\multicolumn{1}{c}{\textbf{Field}} &
	\multicolumn{1}{c}{\textbf{Type}} &
	\multicolumn{1}{c}{\textbf{Null}} &
	\multicolumn{1}{c}{\textbf{Default}} &
	\multicolumn{1}{c}{\textbf{Comments}} \\ \hline \hline
\endhead \endfoot
	\textbf{\textit{id}} & char(36) & Yes & NULL & \parbox[t]{0.35\textwidth}{Identificador de la discusión} \\ \\ \hline
	topic\_id & int(11) & Yes & NULL & \parbox[t]{0.35\textwidth}{Identificador del tema al cual está relacionada la discusión} \\ \\  \hline
	member\_id & int(11) & Yes & NULL & \parbox[t]{0.35\textwidth}{Identificador del usuario que inició la discusión} \\ \\  \hline
	title & varchar(255) & Yes & NULL & \parbox[t]{0.35\textwidth}{Título de la discusión} \\ \\  \hline
	content & text & Yes & NULL & \parbox[t]{0.35\textwidth}{Mensaje} \\ \\  \hline
	sticky & tinyint(1) & Yes & NULL & \parbox[t]{0.35\textwidth}{Discusión fija (se mantiene en el tope de la lista)} \\ \\  \hline
	status & varchar(20) & Yes & unlocked & \parbox[t]{0.35\textwidth}{Estado de la discusión (bloqueada o no)} \\ \\  \hline
	response\_count & int(11) & Yes & NULL & \parbox[t]{0.35\textwidth}{Número de respuestas a la discusión} \\ \\  \hline
	discussion\_visit\_count & int(11) & Yes & NULL & \parbox[t]{0.35\textwidth}{Número de visitas a la discusión} \\ \\  \hline
	created & datetime & Yes & NULL & \parbox[t]{0.35\textwidth}{Fecha de creación de la discusión} \\ \\  \hline
	modified & datetime & Yes & NULL & \parbox[t]{0.35\textwidth}{Fecha de modificación de la discusión} \\ \\ 
\caption[Estructura de la tabla discussions]{Estructura de la tabla discussions. Almacena las discusiones del foro} \label{tab:forum_discussions-structure} \\
\end{longtable}

%
% Structure: forum_responses
%
\begin{longtable}{c c c c l}
	\multicolumn{1}{c}{\textbf{Field}} &
	\multicolumn{1}{c}{\textbf{Type}} &
	\multicolumn{1}{c}{\textbf{Null}} &
	\multicolumn{1}{c}{\textbf{Default}} &
	\multicolumn{1}{c}{\textbf{Comments}} \\ \hline \hline
\endfirsthead
	\multicolumn{1}{c}{\textbf{Field}} &
	\multicolumn{1}{c}{\textbf{Type}} &
	\multicolumn{1}{c}{\textbf{Null}} &
	\multicolumn{1}{c}{\textbf{Default}} &
	\multicolumn{1}{c}{\textbf{Comments}} \\ \hline \hline
\endhead \endfoot
	\textbf{\textit{id}} & char(36) & Yes & NULL & \parbox[t]{0.35\textwidth}{Identificador de la respuesta} \\ \\  \hline
	discussion\_id & char(36) & Yes & NULL & \parbox[t]{0.35\textwidth}{Identificador de la discusión a la cual pertenece la respuesta} \\ \\  \hline
	member\_id & int(11) & Yes & NULL & \parbox[t]{0.35\textwidth}{Identificador del usuario que escribió la respuesta} \\ \\  \hline
	content & text & Yes & NULL & \parbox[t]{0.35\textwidth}{Contenido de la respuesta} \\ \\  \hline
	created & datetime & Yes & NULL & \parbox[t]{0.35\textwidth}{Fecha de creación} \\ \\  \hline
	modified & datetime & Yes & NULL & \parbox[t]{0.35\textwidth}{Fecha de modificación} \\ \\  \hline
\caption[Estructura de la tabla responses]{Estructura de la tabla responses. Almacena las respuestas a las discusiones} \label{tab:forum_responses-structure} \\ 
\end{longtable}

%
% Structure: forum_topics
%
\begin{longtable}{c c c c l}
	\multicolumn{1}{c}{\textbf{Field}} &
	\multicolumn{1}{c}{\textbf{Type}} &
	\multicolumn{1}{c}{\textbf{Null}} &
	\multicolumn{1}{c}{\textbf{Default}} &
	\multicolumn{1}{c}{\textbf{Comments}} \\ \hline \hline
\endfirsthead
	\multicolumn{1}{c}{\textbf{Field}} &
	\multicolumn{1}{c}{\textbf{Type}} &
	\multicolumn{1}{c}{\textbf{Null}} &
	\multicolumn{1}{c}{\textbf{Default}} &
	\multicolumn{1}{c}{\textbf{Comments}} \\ \hline \hline
\endhead \endfoot
	\textbf{\textit{id}} & int(11) & Yes & NULL & \parbox[t]{0.35\textwidth}{Identificador del tema} \\ \\  \hline
	name & varchar(120) & Yes & NULL & \parbox[t]{0.35\textwidth}{Nombre del tema} \\ \\  \hline
	description & varchar(255) & Yes & NULL & \parbox[t]{0.35\textwidth}{Descripción del tema} \\ \\  \hline
	forum\_id & int(11) & Yes & NULL & \parbox[t]{0.35\textwidth}{Identificador del foro al cuál pertenece el tema} \\ \\  \hline
	created & datetime & Yes & NULL & \parbox[t]{0.35\textwidth}{Fecha de creación del tema} \\ \\  \hline
	status & varchar(20) & Yes & unlocked & \parbox[t]{0.35\textwidth}{Estado del tema (bloqueada o no)} \\ \\ 
\caption[Estructura de la tabla topics]{Estructura de la tabla topics. Almacena los temas de los foros} \label{tab:forum_topics-structure} \\
\end{longtable}

\subsection{Tablas relacionadas al plugin Casillero}
%
%
% Structure: locker_documents
%
\begin{longtable}{c c c c l}
	\multicolumn{1}{c}{\textbf{Field}} &
	\multicolumn{1}{c}{\textbf{Type}} &
	\multicolumn{1}{c}{\textbf{Null}} &
	\multicolumn{1}{c}{\textbf{Default}} &
	\multicolumn{1}{c}{\textbf{Comments}} \\ \hline \hline
\endfirsthead
	\multicolumn{1}{c}{\textbf{Field}} &
	\multicolumn{1}{c}{\textbf{Type}} &
	\multicolumn{1}{c}{\textbf{Null}} &
	\multicolumn{1}{c}{\textbf{Default}} &
	\multicolumn{1}{c}{\textbf{Comments}} \\ \hline \hline
\endhead \endfoot
	\textbf{\textit{id}} & int(11) & Yes & NULL & \parbox[t]{0.35\textwidth}{Identificador del documento } \\ \\  \hline
	name & varchar(100) & Yes & NULL & \parbox[t]{0.35\textwidth}{Nombre del documento} \\ \\  \hline
	description & text & Yes & NULL & \parbox[t]{0.35\textwidth}{Descripción del documento} \\ \\  \hline
	file\_name & varchar(150) & Yes & NULL & \parbox[t]{0.35\textwidth}{Nombre del archivo en el disco} \\ \\  \hline
	type & varchar(50) & Yes & application/octet-stream & \parbox[t]{0.35\textwidth}{Tipo MIME del archivo} \\ \\  \hline
	size & int(10) & Yes & NULL & \parbox[t]{0.35\textwidth}{Tamaño del archivo en bytes} \\ \\  \hline
	member\_id & int(11) & Yes & NULL & \parbox[t]{0.35\textwidth}{Dueño del documento} \\ \\ \hline
	folder\_id & char(36) & Yes & NULL & \parbox[t]{0.35\textwidth}{Carpeta contenedora del documento} \\ \\
\caption[Estructura de la tabla documents]{Estructura de la tabla documents. Almacena los datos de los documentos y su ubicación dentro del Casillero} \label{tab:locker_documents-structure} \\
\end{longtable}

%
% Structure: locker_folders
%
\begin{longtable}{c c c c l}
	\multicolumn{1}{c}{\textbf{Field}} &
	\multicolumn{1}{c}{\textbf{Type}} &
	\multicolumn{1}{c}{\textbf{Null}} &
	\multicolumn{1}{c}{\textbf{Default}} &
	\multicolumn{1}{c}{\textbf{Comments}} \\ \hline \hline
\endfirsthead
	\multicolumn{1}{c}{\textbf{Field}} &
	\multicolumn{1}{c}{\textbf{Type}} &
	\multicolumn{1}{c}{\textbf{Null}} &
	\multicolumn{1}{c}{\textbf{Default}} &
	\multicolumn{1}{c}{\textbf{Comments}} \\ \hline \hline
\endhead \endfoot
	\textbf{\textit{id}} & char(36) & Yes &  NULL & \parbox[t]{0.35\textwidth}{Identificador de las carpetas del Casillero} \\ \\ \hline
	name & varchar(100) & Yes & NULL & \parbox[t]{0.35\textwidth}{Nombre de la carpeta} \\ \\ \hline 
	folder\_name & varchar(150) & Yes &  NULL & \parbox[t]{0.35\textwidth}{Nombre de la carpeta?} \\ \\ \hline
	parent\_id & char(36) & Yes & NULL & \parbox[t]{0.35\textwidth}{Identificador de la carpeta padre} \\ \\ \hline
	member\_id & int(11) & Yes & NULL & \parbox[t]{0.35\textwidth}{Dueño de la carpeta} \\ \\
 \caption[Estructura de la tabla folders]{Estructura de la tabla folders. Almacena la información de las carpetas de los Casilleros de cada miembro} \label{tab:locker_folders-structure} \\
\end{longtable}

\subsection{Tablas relacionadas al plugin Quiz}
%
%
% Structure: quiz_choice_choices
%
\begin{longtable}{c c c c l}
	\multicolumn{1}{c}{\textbf{Field}} &
	\multicolumn{1}{c}{\textbf{Type}} &
	\multicolumn{1}{c}{\textbf{Null}} &
	\multicolumn{1}{c}{\textbf{Default}} &
	\multicolumn{1}{c}{\textbf{Comments}} \\ \hline \hline
\endfirsthead
	\multicolumn{1}{c}{\textbf{Field}} &
	\multicolumn{1}{c}{\textbf{Type}} &
	\multicolumn{1}{c}{\textbf{Null}} &
	\multicolumn{1}{c}{\textbf{Default}} &
	\multicolumn{1}{c}{\textbf{Comments}} \\ \hline \hline
\endhead \endfoot
	\textbf{\textit{id}} & char(36) & Yes & NULL & \parbox[t]{0.35\textwidth}{Identificador de la opción} \\ \\  \hline
	choice\_question\_id & char(36) & Yes & NULL & \parbox[t]{0.35\textwidth}{Identificador de la pregunta de selección a la cual pertenece la opción} \\ \\  \hline
	text & text & Yes & NULL & \parbox[t]{0.35\textwidth}{Contenido de la opción} \\ \\  \hline
	position & tinyint(3) & Yes & 0 & \parbox[t]{0.35\textwidth}{Posición fija} \\ \\
\caption[Estructura de la tabla choice\_choices]{Estructura de la tabla choice\_choices. Almacena las opciones de una pregunta de selección} \label{tab:quiz_choice_choices-structure} \\
\end{longtable}

%
% Structure: quiz_choice_questions
%
\begin{longtable}{c c c c l}
	\multicolumn{1}{c}{\textbf{Field}} &
	\multicolumn{1}{c}{\textbf{Type}} &
	\multicolumn{1}{c}{\textbf{Null}} &
	\multicolumn{1}{c}{\textbf{Default}} &
	\multicolumn{1}{c}{\textbf{Comments}} \\ \hline \hline
\endfirsthead
	\multicolumn{1}{c}{\textbf{Field}} &
	\multicolumn{1}{c}{\textbf{Type}} &
	\multicolumn{1}{c}{\textbf{Null}} &
	\multicolumn{1}{c}{\textbf{Default}} &
	\multicolumn{1}{c}{\textbf{Comments}} \\ \hline \hline
\endhead \endfoot
	\textbf{\textit{id}} & char(36) & Yes & NULL & \parbox[t]{0.35\textwidth}{Identificador de la pregunta de selección} \\ \\  \hline
	body & text & Yes & NULL & \parbox[t]{0.35\textwidth}{Planteamiento de la pregunta} \\ \\  \hline
	shuffle & tinyint(1) & Yes & NULL & \parbox[t]{0.35\textwidth}{Determina si deben mostrarse las opciones ordenadas de manera aleatoria} \\ \\  \hline
	max\_choices & int(11) & Yes & NULL & \parbox[t]{0.35\textwidth}{Número máximo de selecciones que puede hacerse como respuesta} \\ \\  \hline
	min\_choices & int(11) & Yes & NULL & \parbox[t]{0.35\textwidth}{Número mínimo de selecciones que debe hacerse como respuesta} \\ \\ 
\caption[Estructura de la tabla choice\_questions]{Estructura de la tabla choice\_questions. Almacena las preguntas de selección} \label{tab:quiz_choice_questions-structure} \\
\end{longtable}

%
% Structure: quiz_choice_questions_quizzes
%
\begin{longtable}{c c c c l}
	\multicolumn{1}{c}{\textbf{Field}} &
	\multicolumn{1}{c}{\textbf{Type}} &
	\multicolumn{1}{c}{\textbf{Null}} &
	\multicolumn{1}{c}{\textbf{Default}} &
	\multicolumn{1}{c}{\textbf{Comments}} \\ \hline \hline
\endfirsthead
	\multicolumn{1}{c}{\textbf{Field}} &
	\multicolumn{1}{c}{\textbf{Type}} &
	\multicolumn{1}{c}{\textbf{Null}} &
	\multicolumn{1}{c}{\textbf{Default}} &
	\multicolumn{1}{c}{\textbf{Comments}} \\ \hline \hline
\endhead \endfoot
	\textbf{\textit{id}} & char(36) & Yes & NULL \\ \hline 
	choice\_question\_id & char(36) & Yes & NULL & \parbox[t]{0.35\textwidth}{Identificador de la pregunta de selección} \\ \\  \hline
	quiz\_id & char(36) & Yes & NULL & \parbox[t]{0.35\textwidth}{Identificador del quiz} \\ \\
\caption[Estructura de la tabla choice\_questions\_quizzes]{Estructura de la tabla choice\_questions\_quizzes. Almacena la asociación entre las preguntas y los quices} \label{tab:quiz_choice_questions_quizzes-structure} \\
\end{longtable}

%
% Structure: quiz_matching_choices
%
\begin{longtable}{c c c c l}
	\multicolumn{1}{c}{\textbf{Field}} &
	\multicolumn{1}{c}{\textbf{Type}} &
	\multicolumn{1}{c}{\textbf{Null}} &
	\multicolumn{1}{c}{\textbf{Default}} &
	\multicolumn{1}{c}{\textbf{Comments}} \\ \hline \hline
\endfirsthead
	\multicolumn{1}{c}{\textbf{Field}} &
	\multicolumn{1}{c}{\textbf{Type}} &
	\multicolumn{1}{c}{\textbf{Null}} &
	\multicolumn{1}{c}{\textbf{Default}} &
	\multicolumn{1}{c}{\textbf{Comments}} \\ \hline \hline
\endhead \endfoot
	\textbf{\textit{id}} & char(36) & Yes & NULL & \parbox[t]{0.35\textwidth}{Identificador de la opción} \\ \\  \hline
	text & text & Yes & NULL  & \parbox[t]{0.35\textwidth}{Contenido de la opción} \\ \\
\caption[Estructura de la tabla matching\_choices]{Estructura de la tabla matching\_choices. Almacena las opciones para las preguntas de apareamiento} \label{tab:quiz_matching_choices-structure} \\
\end{longtable}

%
% Structure: quiz_matching_choices_matching_questions
%
\begin{longtable}{c c c c l}
	\multicolumn{1}{c}{\textbf{Field}} &
	\multicolumn{1}{c}{\textbf{Type}} &
	\multicolumn{1}{c}{\textbf{Null}} &
	\multicolumn{1}{c}{\textbf{Default}} &
	\multicolumn{1}{c}{\textbf{Comments}} \\ \hline \hline
\endfirsthead
	\multicolumn{1}{c}{\textbf{Field}} &
	\multicolumn{1}{c}{\textbf{Type}} &
	\multicolumn{1}{c}{\textbf{Null}} &
	\multicolumn{1}{c}{\textbf{Default}} &
	\multicolumn{1}{c}{\textbf{Comments}} \\ \hline \hline
\endhead \endfoot
	\textbf{\textit{id}} & char(36) & Yes & NULL \\ \hline 
	matching\_question\_id & char(36) & Yes & NULL & \parbox[t]{0.35\textwidth}{Identificador de la pregunta de apareamiento} \\ \\  \hline
	source & char(36) & Yes & NULL & \parbox[t]{0.35\textwidth}{Identificador de la opción fuente} \\ \\  \hline
	target & char(36) & Yes & NULL & \parbox[t]{0.35\textwidth}{Identificador de la opción destino} \\ \\  \hline
	position & tinyint(3) & Yes & 0 & \parbox[t]{0.35\textwidth}{Posición fija de la opción fuente} \\ \\  \hline
\caption[Estructura de la tabla matching\_choices\_matching\_questions]{Estructura de la tabla matching\_choices\_matching\_questions. Almacena la relación entre un par de opciones (pareja correcta) dentro de una pregunta de apareamiento} \label{tab:quiz_matching_choices_matching_questions-structure} \\ 
\end{longtable}

%
% Structure: quiz_matching_questions
%
\begin{longtable}{c c c c l}
	\multicolumn{1}{c}{\textbf{Field}} &
	\multicolumn{1}{c}{\textbf{Type}} &
	\multicolumn{1}{c}{\textbf{Null}} &
	\multicolumn{1}{c}{\textbf{Default}} &
	\multicolumn{1}{c}{\textbf{Comments}} \\ \hline \hline
\endfirsthead
	\multicolumn{1}{c}{\textbf{Field}} &
	\multicolumn{1}{c}{\textbf{Type}} &
	\multicolumn{1}{c}{\textbf{Null}} &
	\multicolumn{1}{c}{\textbf{Default}} &
	\multicolumn{1}{c}{\textbf{Comments}} \\ \hline \hline
\endhead \endfoot
	\textbf{\textit{id}} & char(36) & Yes & NULL & \parbox[t]{0.35\textwidth}{Identificador de la pregunta de apareamiento} \\ \\  \hline
	body & text & Yes & NULL & \parbox[t]{0.35\textwidth}{Planteamiento de la pregunta} \\ \\  \hline
	shuffle & tinyint(1) & Yes & NULL & \parbox[t]{0.35\textwidth}{Determina si deben mostrarse las opciones ordenadas de manera aleatoria} \\ \\  \hline
	max\_associations & int(11) & Yes & NULL & \parbox[t]{0.35\textwidth}{Número máximo de emparejamientos que se pueden hacer para una respuesta} \\ \\  \hline
	min\_associations & int(11) & Yes & NULL & \parbox[t]{0.35\textwidth}{Número mínimo de emparejamientos que deben realizarse para una respuesta} \\ \\ 
\caption{Estructura de la tabla matching\_questions} \label{tab:quiz_matching_questions-structure} \\ 
\end{longtable}

%
% Structure: quiz_matching_questions_quizzes
%
\begin{longtable}{c c c c l}
	\multicolumn{1}{c}{\textbf{Field}} &
	\multicolumn{1}{c}{\textbf{Type}} &
	\multicolumn{1}{c}{\textbf{Null}} &
	\multicolumn{1}{c}{\textbf{Default}} &
	\multicolumn{1}{c}{\textbf{Comments}} \\ \hline \hline
\endfirsthead
	\multicolumn{1}{c}{\textbf{Field}} &
	\multicolumn{1}{c}{\textbf{Type}} &
	\multicolumn{1}{c}{\textbf{Null}} &
	\multicolumn{1}{c}{\textbf{Default}} &
	\multicolumn{1}{c}{\textbf{Comments}} \\ \hline \hline
\endhead \endfoot
	\textbf{\textit{id}} & char(36) & Yes & NULL \\ \hline 
	matching\_question\_id & char(36) & Yes & NULL & \parbox[t]{0.35\textwidth}{Identificador de la pregunta de apareamiento} \\ \\  \hline
	quiz\_id & char(36) & Yes & NULL & \parbox[t]{0.35\textwidth}{Identificador del quiz} \\ \\ 
\caption[Estructura de la tabla matching\_questions\_quizzes]{Estructura de la tabla matching\_questions\_quizzes. Mantiene las preguntas de apareamiento por quiz} \label{tab:quiz_matching_questions_quizzes-structure} \\
\end{longtable}

%
% Structure: quiz_ordering_choices
%
\begin{longtable}{c c c c l}
	\multicolumn{1}{c}{\textbf{Field}} &
	\multicolumn{1}{c}{\textbf{Type}} &
	\multicolumn{1}{c}{\textbf{Null}} &
	\multicolumn{1}{c}{\textbf{Default}} &
	\multicolumn{1}{c}{\textbf{Comments}} \\ \hline \hline
\endfirsthead
	\multicolumn{1}{c}{\textbf{Field}} &
	\multicolumn{1}{c}{\textbf{Type}} &
	\multicolumn{1}{c}{\textbf{Null}} &
	\multicolumn{1}{c}{\textbf{Default}} &
	\multicolumn{1}{c}{\textbf{Comments}} \\ \hline \hline
\endhead \endfoot
	\textbf{\textit{id}} & char(36) & Yes & NULL & \parbox[t]{0.35\textwidth}{Identificador de la opción} \\ \\  \hline
	ordering\_question\_id & char(36) & Yes & NULL  & \parbox[t]{0.35\textwidth}{Identificador de la pregunta de ordenamiento} \\ \\  \hline
	text & text & Yes & NULL  & \parbox[t]{0.35\textwidth}{Contenido de la opción} \\ \\  \hline
	position & tinyint(3) & Yes & 0  & \parbox[t]{0.35\textwidth}{Posición fija de la opción} \\ \\
\caption[Estructura de la tabla ordering\_choices]{Estructura de la tabla ordering\_choices. Almacena las opciones para las preguntas de ordenamiento} \label{tab:quiz_ordering_choices-structure} \\
\end{longtable}

%
% Structure: quiz_ordering_questions
%
\begin{longtable}{c c c c l}
	\multicolumn{1}{c}{\textbf{Field}} &
	\multicolumn{1}{c}{\textbf{Type}} &
	\multicolumn{1}{c}{\textbf{Null}} &
	\multicolumn{1}{c}{\textbf{Default}} &
	\multicolumn{1}{c}{\textbf{Comments}} \\ \hline \hline
\endfirsthead
	\multicolumn{1}{c}{\textbf{Field}} &
	\multicolumn{1}{c}{\textbf{Type}} &
	\multicolumn{1}{c}{\textbf{Null}} &
	\multicolumn{1}{c}{\textbf{Default}} &
	\multicolumn{1}{c}{\textbf{Comments}} \\ \hline \hline
\endhead \endfoot
	\textbf{\textit{id}} & char(36) & Yes & NULL & \parbox[t]{0.35\textwidth}{Identificador de la pregunta de ordenamiento} \\ \\  \hline
	body & text & Yes & NULL & \parbox[t]{0.35\textwidth}{Planteamiento de la pregunta} \\ \\  \hline
	shuffle & tinyint(1) & Yes & NULL & \parbox[t]{0.35\textwidth}{Determina si deben mostrarse las opciones ordenadas de manera aleatoria} \\ \\  \hline
	max\_choices & int(11) & Yes & NULL & \parbox[t]{0.35\textwidth}{Número máximo de opciones que pueden ser parte de la respuesta} \\ \\  \hline
	min\_choices & int(11) & Yes & NULL & \parbox[t]{0.35\textwidth}{Número mínimo de opciones que deben ser parte de la respuesta} \\ \\  \hline
 \caption[Estructura de la tabla ordering\_questions]{Estructura de la tabla ordering\_questions. Almacena las preguntas de ordenamiento} \label{tab:quiz_ordering_questions-structure} \\
\end{longtable}

%
% Structure: quiz_ordering_questions_quizzes
%
\begin{longtable}{c c c c l}
	\multicolumn{1}{c}{\textbf{Field}} &
	\multicolumn{1}{c}{\textbf{Type}} &
	\multicolumn{1}{c}{\textbf{Null}} &
	\multicolumn{1}{c}{\textbf{Default}} &
	\multicolumn{1}{c}{\textbf{Comments}} \\ \hline \hline
\endfirsthead
	\multicolumn{1}{c}{\textbf{Field}} &
	\multicolumn{1}{c}{\textbf{Type}} &
	\multicolumn{1}{c}{\textbf{Null}} &
	\multicolumn{1}{c}{\textbf{Default}} &
	\multicolumn{1}{c}{\textbf{Comments}} \\ \hline \hline
\endhead \endfoot
	\textbf{\textit{id}} & char(36) & Yes & NULL \\ \hline 
	ordering\_question\_id & char(36) & Yes & NULL & \parbox[t]{0.35\textwidth}{Identificador de la pregunta de ordenamiento} \\ \\  \hline
	quiz\_id & char(36) & Yes & NULL & \parbox[t]{0.35\textwidth}{Identificador del quiz} \\ \\
\caption[Estructura de la tabla ordering\_questions\_quizzes]{Estructura de la tabla ordering\_questions\_quizzes. Mantiene la asociación entre quices y las pregunta de ordenamiento} \label{tab:quiz_ordering_questions_quizzes-structure} \\
\end{longtable}

%
% Structure: quiz_quizzes
%
\begin{longtable}{c c c c l}
	\multicolumn{1}{c}{\textbf{Field}} &
	\multicolumn{1}{c}{\textbf{Type}} &
	\multicolumn{1}{c}{\textbf{Null}} &
	\multicolumn{1}{c}{\textbf{Default}} &
	\multicolumn{1}{c}{\textbf{Comments}} \\ \hline \hline
\endfirsthead
	\multicolumn{1}{c}{\textbf{Field}} &
	\multicolumn{1}{c}{\textbf{Type}} &
	\multicolumn{1}{c}{\textbf{Null}} &
	\multicolumn{1}{c}{\textbf{Default}} &
	\multicolumn{1}{c}{\textbf{Comments}} \\ \hline \hline
\endhead \endfoot
	\textbf{\textit{id}} & char(36) & Yes & NULL & \parbox[t]{0.35\textwidth}{Identificador del quiz} \\ \\ \hline
	name & varchar(255) & Yes & NULL & \parbox[t]{0.35\textwidth}{Nombre o título del Quiz} \\ \\
\caption[Estructura de la tabla quizzes]{Estructura de la tabla quizzes. Almacena los quices registrados} \label{tab:quiz_quizzes-structure} \\
\end{longtable}

%
% Structure: quiz_text_questions
%
\begin{longtable}{c c c c l}
	\multicolumn{1}{c}{\textbf{Field}} &
	\multicolumn{1}{c}{\textbf{Type}} &
	\multicolumn{1}{c}{\textbf{Null}} &
	\multicolumn{1}{c}{\textbf{Default}} &
	\multicolumn{1}{c}{\textbf{Comments}} \\ \hline \hline
\endfirsthead
	\multicolumn{1}{c}{\textbf{Field}} &
	\multicolumn{1}{c}{\textbf{Type}} &
	\multicolumn{1}{c}{\textbf{Null}} &
	\multicolumn{1}{c}{\textbf{Default}} &
	\multicolumn{1}{c}{\textbf{Comments}} \\ \hline \hline
\endhead \endfoot
	\textbf{\textit{id}} & char(36) & Yes & NULL & \parbox[t]{0.35\textwidth}{Identificador de la pregunta} \\ \\ \hline
	title & varchar(255) & Yes & NULL & \parbox[t]{0.35\textwidth}{Título de la pregunta} \\ \\ \hline
	body & text & Yes & NULL  & \parbox[t]{0.35\textwidth}{Planteamiento de la pregunta} \\ \\ \hline
	format & varchar(5) & Yes & NULL  & \parbox[t]{0.35\textwidth}{Formato de la respuesta} \\ \\
\caption[Estructura de la tabla text\_questions]{Estructura de la tabla text\_questions. Almacena las preguntas de desarollo} \label{tab:quiz_text_questions-structure} \\
\end{longtable}

%
% Structure: quiz_text_questions_quizzes
%
\begin{longtable}{c c c c l}
	\multicolumn{1}{c}{\textbf{Field}} &
	\multicolumn{1}{c}{\textbf{Type}} &
	\multicolumn{1}{c}{\textbf{Null}} &
	\multicolumn{1}{c}{\textbf{Default}} &
	\multicolumn{1}{c}{\textbf{Comments}} \\ \hline \hline
\endfirsthead
	\multicolumn{1}{c}{\textbf{Field}} &
	\multicolumn{1}{c}{\textbf{Type}} &
	\multicolumn{1}{c}{\textbf{Null}} &
	\multicolumn{1}{c}{\textbf{Default}} &
	\multicolumn{1}{c}{\textbf{Comments}} \\ \hline \hline
\endhead \endfoot
	\textbf{\textit{id}} & char(36) & Yes & NULL & \parbox[t]{0.35\textwidth}{Identificador de las preguntas de los quizzes}\\ \hline 
	\textbf{text\_question\_id} & char(36) & Yes & NULL & \parbox[t]{0.35\textwidth}{Identificador de la pregunta de desarollo} \\ \\ \hline 
	\textbf{quiz\_id} & char(36) & Yes & NULL & \parbox[t]{0.35\textwidth}{Identificador del quiz} \\ \\
\caption[Estructura de la tabla text\_questions\_quizzes]{Estructura de la tabla text\_questions\_quizzes. Mantiene la asociación entre quices y las pregunta de desarollo} \label{tab:quiz_text_questions_quizzes-structure} \\
\end{longtable}


\subsection{Tablas relacionadas al plugin Scorm}
%
% Structure: scorm_attendee_trackings
%
\begin{longtable}{c c c c l}
	\multicolumn{1}{c}{\textbf{Field}} &
	\multicolumn{1}{c}{\textbf{Type}} &
	\multicolumn{1}{c}{\textbf{Null}} &
	\multicolumn{1}{c}{\textbf{Default}} &
	\multicolumn{1}{c}{\textbf{Comments}} \\ \hline \hline
\endfirsthead
	\multicolumn{1}{c}{\textbf{Field}} &
	\multicolumn{1}{c}{\textbf{Type}} &
	\multicolumn{1}{c}{\textbf{Null}} &
	\multicolumn{1}{c}{\textbf{Default}} &
	\multicolumn{1}{c}{\textbf{Comments}} \\ \hline \hline
\endhead \endfoot
	sco\_id & int(11) & Yes & NULL & \parbox[t]{0.35\textwidth}{Identificador del SCO al cual está asociado attendee trackings}\\ \hline 
	student\_id & int(11) & Yes & NULL & \parbox[t]{0.35\textwidth}{Identificador del estudiante al cual está asociado attendee trackings}\\ \hline 
	datamodel\_element & varchar(255) & Yes & NULL \\ \hline 
	value & varchar(255) & Yes & NULL \\ \\ 
\caption{Estructura de la tabla scorm\_attendee\_trackings} \label{tab:scorm_attendee_trackings-structure} \\
\end{longtable}

%
% Structure: scorm_choice_considerations
%
\begin{longtable}{c c c c l}
	\multicolumn{1}{c}{\textbf{Field}} &
	\multicolumn{1}{c}{\textbf{Type}} &
	\multicolumn{1}{c}{\textbf{Null}} &
	\multicolumn{1}{c}{\textbf{Default}} &
	\multicolumn{1}{c}{\textbf{Comments}} \\ \hline \hline
\endfirsthead
	\multicolumn{1}{c}{\textbf{Field}} &
	\multicolumn{1}{c}{\textbf{Type}} &
	\multicolumn{1}{c}{\textbf{Null}} &
	\multicolumn{1}{c}{\textbf{Default}} &
	\multicolumn{1}{c}{\textbf{Comments}} \\ \hline \hline
\endhead \endfoot
	\textbf{\textit{id}} & int(11) & Yes & NULL \\ \hline 
	sco\_id & int(11) & Yes & NULL & \parbox[t]{0.35\textwidth}{Identificador del SCO al cual está asociado choice considerations}\\ \hline 
	preventActivation & varchar(5) & Yes & false \\ \hline 
	constrainChoice & varchar(5) & Yes & false \\ \\ 
\caption{Estructura de la tabla scorm\_choice\_considerations} \label{tab:scorm_choice_considerations-structure} \\ 
\end{longtable}

%
% Structure: scorm_conditions
%
\begin{longtable}{c c c c l}
	\multicolumn{1}{c}{\textbf{Field}} &
	\multicolumn{1}{c}{\textbf{Type}} &
	\multicolumn{1}{c}{\textbf{Null}} &
	\multicolumn{1}{c}{\textbf{Default}} &
	\multicolumn{1}{c}{\textbf{Comments}} \\ \hline \hline
\endfirsthead
	\multicolumn{1}{c}{\textbf{Field}} &
	\multicolumn{1}{c}{\textbf{Type}} &
	\multicolumn{1}{c}{\textbf{Null}} &
	\multicolumn{1}{c}{\textbf{Default}} &
	\multicolumn{1}{c}{\textbf{Comments}} \\ \hline \hline
\endhead \endfoot
	\textbf{\textit{id}} & int(11) & Yes & NULL\\ \hline 
	referencedObjective & varchar(255) & Yes & NULL & \parbox[t]{0.35\textwidth}{Corresponde al atributo referencedObjective del elemento ruleCondition de SCORM} \\ \\  \hline
	measureThreshold & varchar(7) & Yes & NULL & \parbox[t]{0.35\textwidth}{Corresponde al atributo measureThreshold del elemento ruleCondition de SCORM} \\ \\  \hline
	operator & varchar(4) & Yes & noOp & \parbox[t]{0.35\textwidth}{Corresponde al atributo operator del elemento ruleCondition de SCORM} \\ \\  \hline
	ruleCondition & varchar(27) & Yes & NULL \\ \hline 
	rule\_id & int(11) & Yes & NULL & \parbox[t]{0.35\textwidth}{Identificador de la regla a la cual está asociada condition}\\ \\ 
\caption{Estructura de la tabla conditions} \label{tab:scorm_conditions-structure} \\
\end{longtable}

%
% Structure: scorm_control_modes
%
\begin{longtable}{c c c c l}
	\multicolumn{1}{c}{\textbf{Field}} &
	\multicolumn{1}{c}{\textbf{Type}} &
	\multicolumn{1}{c}{\textbf{Null}} &
	\multicolumn{1}{c}{\textbf{Default}} &
	\multicolumn{1}{c}{\textbf{Comments}} \\ \hline \hline
\endfirsthead
	\multicolumn{1}{c}{\textbf{Field}} &
	\multicolumn{1}{c}{\textbf{Type}} &
	\multicolumn{1}{c}{\textbf{Null}} &
	\multicolumn{1}{c}{\textbf{Default}} &
	\multicolumn{1}{c}{\textbf{Comments}} \\ \hline \hline
\endhead \endfoot
	\textbf{\textit{id}} & int(11) & Yes & NULL \\ \hline 
	sco\_id & int(11) & Yes & NULL & \parbox[t]{0.35\textwidth}{Identificador del SCO al cual está asociado control modes}\\ \hline 
	choiceExit & varchar(5) & Yes & true & \parbox[t]{0.35\textwidth}{Corresponde al atributo choiceExit del elemento controlMode de SCORM} \\ \\  \hline
	choice & varchar(5) & Yes & true & \parbox[t]{0.35\textwidth}{Corresponde al atributo choice del elemento controlMode de SCORM } \\ \\  \hline
	flow & varchar(5) & Yes & false & \parbox[t]{0.35\textwidth}{Corresponde al atributo flow del elemento controlMode de SCORM} \\ \\  \hline
	forwardOnly & varchar(5) & Yes & false & \parbox[t]{0.35\textwidth}{Corresponde al atributo forwardOnly del elemento controlMode de SCORM} \\ \\  \hline
	useCurrentAttemptObjectiveInfo & varchar(5) & Yes & true & \parbox[t]{0.35\textwidth}{Corresponde al atributo useCurrentAttemptObjectiveInfo del elemento controlMode de SCORM} \\ \\  \hline
	useCurrentAttemptProgressInfo & varchar(5) & Yes & true & \parbox[t]{0.35\textwidth}{Corresponde al atributo useCurrentAttemptProgressInfo del elemento controlMode de SCORM} \\ \\  \hline
 \caption{Estructura de la tabla control\_modes} \label{tab:scorm_control_modes-structure} \\
\end{longtable}

%
% Structure: scorm_delivery_controls
%
\begin{longtable}{c c c c l}
	\multicolumn{1}{c}{\textbf{Field}} &
	\multicolumn{1}{c}{\textbf{Type}} &
	\multicolumn{1}{c}{\textbf{Null}} &
	\multicolumn{1}{c}{\textbf{Default}} &
	\multicolumn{1}{c}{\textbf{Comments}} \\ \hline \hline
\endfirsthead
	\multicolumn{1}{c}{\textbf{Field}} &
	\multicolumn{1}{c}{\textbf{Type}} &
	\multicolumn{1}{c}{\textbf{Null}} &
	\multicolumn{1}{c}{\textbf{Default}} &
	\multicolumn{1}{c}{\textbf{Comments}} \\ \hline \hline
\endhead \endfoot
	\textbf{\textit{id}} & int(11) & Yes & NULL & \parbox[t]{0.35\textwidth}{Identificador de delivery control}\\ \hline 
	sco\_id & int(11) & Yes & NULL & \parbox[t]{0.35\textwidth}{Identificador del SCO al cual está asociado delivery controls}\\ \hline 
	tracked & varchar(5) & Yes & true & \parbox[t]{0.35\textwidth}{Corresponde al atributo tracked del elemento deliveryControls de SCORM} \\ \\  \hline
	completionSetByContent & varchar(5) & Yes & false & \parbox[t]{0.35\textwidth}{Corresponde al atributo completionSetByContent del elemento deliveryControls de SCORM} \\ \\  \hline
	objectiveSetByContent & varchar(5) & Yes & false & \parbox[t]{0.35\textwidth}{Corresponde al atributo objectiveSetByContent  del elemento deliveryControls de SCORM} \\ \\  \hline
 \caption{Estructura de la tabla delivery\_controls} \label{tab:scorm_delivery_controls-structure} \\
\end{longtable}

%
% Structure: scorm_map_infos
%
\begin{longtable}{c c c c l}
	\multicolumn{1}{c}{\textbf{Field}} &
	\multicolumn{1}{c}{\textbf{Type}} &
	\multicolumn{1}{c}{\textbf{Null}} &
	\multicolumn{1}{c}{\textbf{Default}} &
	\multicolumn{1}{c}{\textbf{Comments}} \\ \hline \hline
\endfirsthead
	\multicolumn{1}{c}{\textbf{Field}} &
	\multicolumn{1}{c}{\textbf{Type}} &
	\multicolumn{1}{c}{\textbf{Null}} &
	\multicolumn{1}{c}{\textbf{Default}} &
	\multicolumn{1}{c}{\textbf{Comments}} \\ \hline \hline
\endhead \endfoot
	\textbf{\textit{id}} & int(11)  & Yes & NULL & \parbox[t]{0.35\textwidth}{Identificador de map info}l\\ \hline 
	objective\_id & int(11) & Yes & NULL & \parbox[t]{0.35\textwidth}{Identificador del objective al cual está asociado map info}\\ \hline 
	targetObjectiveID & varchar(255) & Yes & NULL & \parbox[t]{0.35\textwidth}{Corresponde al atributo targetObjectiveID del elemento mapInfo de SCORM} \\ \\  \hline
	readSatisfiedStatus & varchar(5) & Yes & true & \parbox[t]{0.35\textwidth}{Corresponde al atributo readSatisfiedStatus del elemento mapInfo de SCORM} \\ \\  \hline
	readNormalizedMeasure & varchar(5) & Yes & true & \parbox[t]{0.35\textwidth}{Corresponde al atributo readNormalizedMeasure del elemento mapInfo de SCORM} \\ \\  \hline
	writeSatisfiedStatus & varchar(5) & Yes & false & \parbox[t]{0.35\textwidth}{Corresponde al atributo writeSatisfiedStatus del elemento mapInfo de SCORM} \\ \\  \hline
	writeNormalizedMeasure & varchar(5) & Yes & false & \parbox[t]{0.35\textwidth}{Corresponde al atributo writeNormalizedMeasure del elemento mapInfo de SCORM} \\ \\  \hline
 \caption{Estructura de la tabla map\_infos} \label{tab:scorm_map_infos-structure} \\
\end{longtable}

%
% Structure: scorm_objectives
%
\begin{longtable}{c c c c l}
	\multicolumn{1}{c}{\textbf{Field}} &
	\multicolumn{1}{c}{\textbf{Type}} &
	\multicolumn{1}{c}{\textbf{Null}} &
	\multicolumn{1}{c}{\textbf{Default}} &
	\multicolumn{1}{c}{\textbf{Comments}} \\ \hline \hline
\endfirsthead
	\multicolumn{1}{c}{\textbf{Field}} &
	\multicolumn{1}{c}{\textbf{Type}} &
	\multicolumn{1}{c}{\textbf{Null}} &
	\multicolumn{1}{c}{\textbf{Default}} &
	\multicolumn{1}{c}{\textbf{Comments}} \\ \hline \hline
\endhead \endfoot
	\textbf{\textit{id}} & int(11) & Yes & NULL & \parbox[t]{0.35\textwidth}{Identificador de objectives}\\ \hline 
	sco\_id & int(11) & Yes & NULL & \parbox[t]{0.35\textwidth}{Identificador del SCO al cual está asociado objectives}\\ \hline 
	satisfiedByMeasure & varchar(5) & Yes & false & \parbox[t]{0.35\textwidth}{Corresponde al atributo satisfiedByMeasure del elemento objective de SCORM} \\ \\  \hline
	minNormalizedMeasure & varchar(3) & Yes & 1.0 & \parbox[t]{0.35\textwidth}{Corresponde al elemento minNormalizedMeasure del elemento objective de SCORM} \\ \\  \hline
	objectiveID & varchar(255) & Yes & NULL & \parbox[t]{0.35\textwidth}{Corresponde al atributo objectiveID del elemento objective de SCORM} \\ \\  \hline
	primary & tinyint(1) & Yes & 0 \\ \\ 
 \caption{Estructura de la tabla objectives} \label{tab:scorm_objectives-structure} \\
\end{longtable}

%
% Structure: scorm_randomizations
%
\begin{longtable}{c c c c l}
	\multicolumn{1}{c}{\textbf{Field}} &
	\multicolumn{1}{c}{\textbf{Type}} &
	\multicolumn{1}{c}{\textbf{Null}} &
	\multicolumn{1}{c}{\textbf{Default}} &
	\multicolumn{1}{c}{\textbf{Comments}} \\ \hline \hline
\endfirsthead
	\multicolumn{1}{c}{\textbf{Field}} &
	\multicolumn{1}{c}{\textbf{Type}} &
	\multicolumn{1}{c}{\textbf{Null}} &
	\multicolumn{1}{c}{\textbf{Default}} &
	\multicolumn{1}{c}{\textbf{Comments}} \\ \hline \hline
\endhead \endfoot
	\textbf{\textit{id}} & int(11)  & Yes & NULL & \parbox[t]{0.35\textwidth}{Identificador de randomization}\\ \hline 
	sco\_id & int(11) & Yes & NULL & \parbox[t]{0.35\textwidth}{Identificador del SCO al cual está asociado el randomization}\\ \hline 
	randomizationTiming & varchar(16) & Yes & never & \parbox[t]{0.35\textwidth}{Corresponde al atributo randomizationTiming del elemento randomizationControls de SCORM } \\ \\  \hline
	selectCount & int(11)  & Yes & NULL & \parbox[t]{0.35\textwidth}{Corresponde al atributo randomizationTiming del elemento randomizationControls de SCORM} \\ \\  \hline
	reorderChildren & varchar(5) & Yes & false & \parbox[t]{0.35\textwidth}{Corresponde al atributo reorderChildren del elemento randomizationControls de SCORM} \\ \\  \hline
	selectionTiming & varchar(16) & Yes & never & \parbox[t]{0.35\textwidth}{Corresponde al atributo selectionTiming del elemento randomizationControls de SCORM} \\ \\  \hline
 \caption{Estructura de la tabla randomizations} \label{tab:scorm_randomizations-structure} \\
\end{longtable}

%
% Structure: scorm_rollups
%
\begin{longtable}{c c c c l}
	\multicolumn{1}{c}{\textbf{Field}} &
	\multicolumn{1}{c}{\textbf{Type}} &
	\multicolumn{1}{c}{\textbf{Null}} &
	\multicolumn{1}{c}{\textbf{Default}} &
	\multicolumn{1}{c}{\textbf{Comments}} \\ \hline \hline
\endfirsthead
	\multicolumn{1}{c}{\textbf{Field}} &
	\multicolumn{1}{c}{\textbf{Type}} &
	\multicolumn{1}{c}{\textbf{Null}} &
	\multicolumn{1}{c}{\textbf{Default}} &
	\multicolumn{1}{c}{\textbf{Comments}} \\ \hline \hline
\endhead \endfoot
	\textbf{\textit{id}} & int(11) & Yes & NULL & \parbox[t]{0.35\textwidth}{Identificador del rollup}\\ \hline 
	sco\_id & int(11) & Yes & NULL & \parbox[t]{0.35\textwidth}{Identificador del SCO al cual está asociado rollup}\\ \hline 
	rollupObjectiveSatisfied & varchar(5) & Yes & true & \parbox[t]{0.35\textwidth}{Corresponde al atributo rollupObjectiveSatisfied del elemento rollupRules de SCORM} \\ \\  \hline
	rollupProgressCompletion & varchar(5) & Yes & true & \parbox[t]{0.35\textwidth}{Corresponde al atributo rollupProgressCompletion del elemento rollupRules de SCORM} \\ \\  \hline
	objectiveMeasureWeight & varchar(20) & Yes & 1.0000 & \parbox[t]{0.35\textwidth}{Corresponde al atributo objectiveMeasureWeight del elemento rollupRules de SCORM} \\ \\  \hline
 \caption{Estructura de la tabla rollups} \label{tab:scorm_rollups-structure} \\
\end{longtable}

%
% Structure: scorm_rollup_considerations
%
\begin{longtable}{c c c c l}
	\multicolumn{1}{c}{\textbf{Field}} &
	\multicolumn{1}{c}{\textbf{Type}} &
	\multicolumn{1}{c}{\textbf{Null}} &
	\multicolumn{1}{c}{\textbf{Default}} &
	\multicolumn{1}{c}{\textbf{Comments}} \\ \hline \hline
\endfirsthead
	\multicolumn{1}{c}{\textbf{Field}} &
	\multicolumn{1}{c}{\textbf{Type}} &
	\multicolumn{1}{c}{\textbf{Null}} &
	\multicolumn{1}{c}{\textbf{Default}} &
	\multicolumn{1}{c}{\textbf{Comments}} \\ \hline \hline
\endhead \endfoot
	\textbf{\textit{id}} & int(11) & Yes & NULL \\ \hline 
	sco\_id & int(11) & Yes & NULL & \parbox[t]{0.35\textwidth}{Identificador del SCO al cual está asociado el rollup consideration}\\ \hline 
	requiredForSatisfied & varchar(15) & Yes & always & \parbox[t]{0.35\textwidth}{Corresponde al atributo requiredForSatisfied del elemento rollupConsiderations de SCORM } \\ \\  \hline
	requiredForNotSatisfied & varchar(15) & Yes & always & \parbox[t]{0.35\textwidth}{Corresponde al atributo requiredForNotSatisfied del elemento rollupConsiderations de SCORM} \\ \\  \hline
	requiredForComplete & varchar(15) & Yes & always & \parbox[t]{0.35\textwidth}{Corresponde al atributo requiredForComplete del elemento rollupConsiderations de SCORM} \\ \\  \hline
	requiredForIncomplete & varchar(15) & Yes & always & \parbox[t]{0.35\textwidth}{Corresponde al atributo requiredForIncomplete del elemento rollupConsiderations de SCORM} \\ \\  \hline
	measureSatisfactionIfActive & varchar(5) & Yes & true & \parbox[t]{0.35\textwidth}{Corresponde al atributo measureSatisfactionIfActive del elemento rollupConsiderations de SCORM} \\ \\  \hline
 \caption{Estructura de la tabla rollup\_considerations} \label{tab:scorm_rollup_considerations-structure} \\
\end{longtable}

%
% Structure: scorm_rules
%
\begin{longtable}{c c c c l}
	\multicolumn{1}{c}{\textbf{Field}} &
	\multicolumn{1}{c}{\textbf{Type}} &
	\multicolumn{1}{c}{\textbf{Null}} &
	\multicolumn{1}{c}{\textbf{Default}} &
	\multicolumn{1}{c}{\textbf{Comments}} \\ \hline \hline
\endfirsthead
	\multicolumn{1}{c}{\textbf{Field}} &
	\multicolumn{1}{c}{\textbf{Type}} &
	\multicolumn{1}{c}{\textbf{Null}} &
	\multicolumn{1}{c}{\textbf{Default}} &
	\multicolumn{1}{c}{\textbf{Comments}} \\ \hline \hline
\endhead \endfoot
	\textbf{\textit{id}} & int(11) & Yes & NULL \\ \hline 
	sco\_id & int(11) & Yes & NULL & \parbox[t]{0.35\textwidth}{Identificador del SCO al cual está asociado rules}\\ \hline 
	type & varchar(4) & Yes & NULL \\ \hline 
	conditionCombination & varchar(3) & Yes & all & \parbox[t]{0.35\textwidth}{Corresponde al atributo conditionCombination del elemento ruleConditions de SCORM}\\ \hline 
	action & varchar(20) & Yes & NULL \\ \hline 
	minimumPercent & varchar(6) & Yes & 0.0000 & \parbox[t]{0.35\textwidth}{Corresponde al atributo minimumPercent del elemento rollupRule de SCORM}\\ \hline 
	minimumCount & varchar(5) & Yes & 0 & \parbox[t]{0.35\textwidth}{Corresponde al atributo minimumCount del elemento rollupRule de SCORM} \\ \hline 
	rollup\_id & int(11) & Yes & NULL \\ \\ 
 \caption{Estructura de la tabla rules} \label{tab:scorm_rules-structure} \\
\end{longtable}

%
% Structure: scorm_scorms
%
\begin{longtable}{c c c c l}
	\multicolumn{1}{c}{\textbf{Field}} &
	\multicolumn{1}{c}{\textbf{Type}} &
	\multicolumn{1}{c}{\textbf{Null}} &
	\multicolumn{1}{c}{\textbf{Default}} &
	\multicolumn{1}{c}{\textbf{Comments}} \\ \hline \hline
\endfirsthead
	\multicolumn{1}{c}{\textbf{Field}} &
	\multicolumn{1}{c}{\textbf{Type}} &
	\multicolumn{1}{c}{\textbf{Null}} &
	\multicolumn{1}{c}{\textbf{Default}} &
	\multicolumn{1}{c}{\textbf{Comments}} \\ \hline \hline
\endhead \endfoot
	\textbf{\textit{id}} & int(11)  & Yes & NULL & \parbox[t]{0.35\textwidth}{Identificador del SCORM} \\ \hline 
	course\_id & int(11) & Yes & NULL & \parbox[t]{0.35\textwidth}{Identificador del curso en el cual se encuentra el SCORM}\\ \hline 
	name & varchar(255) & Yes & NULL & \parbox[t]{0.35\textwidth}{Nombre del recurso SCORM}\\ \hline 
	file\_name & varchar(255) & Yes & NULL & \parbox[t]{0.35\textwidth}{Nombre del archivo que contiene el paquete SCORM}\\ \hline 
	description & text & Yes & NULL & \parbox[t]{0.35\textwidth}{Descripción del recurso SCORM}\\ \hline 
	version & varchar(9) & Yes & NULL & \parbox[t]{0.35\textwidth}{Version del SCORM}\\ \hline 
	created & datetime & Yes & NULL & \parbox[t]{0.35\textwidth}{Fecha de creación del SCORM}\\ \hline 
	modified & datetime & Yes & NULL & \parbox[t]{0.35\textwidth}{Fecha de modificación del SCORM}\\ \hline 
	path & text & Yes & NULL \\ \\ 
 \caption{Estructura de la tabla scorms} \label{tab:scorm_scorms-structure} \\
\end{longtable}

%
% Structure: scorm_scos
%
\begin{longtable}{c c c c l}
	\multicolumn{1}{c}{\textbf{Field}} &
	\multicolumn{1}{c}{\textbf{Type}} &
	\multicolumn{1}{c}{\textbf{Null}} &
	\multicolumn{1}{c}{\textbf{Default}} &
	\multicolumn{1}{c}{\textbf{Comments}} \\ \hline \hline
\endfirsthead
	\multicolumn{1}{c}{\textbf{Field}} &
	\multicolumn{1}{c}{\textbf{Type}} &
	\multicolumn{1}{c}{\textbf{Null}} &
	\multicolumn{1}{c}{\textbf{Default}} &
	\multicolumn{1}{c}{\textbf{Comments}} \\ \hline \hline
\endhead \endfoot
	\textbf{\textit{id}} & int(11)  & Yes & NULL & \parbox[t]{0.35\textwidth}{Identificador del SCO} \\ \hline 
	scorm\_id & int(11)  & Yes & NULL & \parbox[t]{0.35\textwidth}{Identificador del paquete SCORM al cual pertenece el SCO} \\ \hline 
	parent\_id & int(11)  & Yes & NULL & \parbox[t]{0.35\textwidth}{Identificador del padre del SCO}\\ \hline 
	manifest & varchar(255) & Yes & NULL \\ \hline 
	organization & varchar(255) & Yes & NULL \\ \hline 
	identifier & varchar(255) & Yes & NULL & \parbox[t]{0.35\textwidth}{Corresponde al atributo identifier del elemento item de SCORM} \\ \\  \hline
	href & varchar(255) & Yes & NULL \\ \hline 
	title & varchar(255) & Yes & NULL & \parbox[t]{0.35\textwidth}{Título del SCO} \\ \hline 
	completionThreshold & varchar(3) & Yes & NULL & \parbox[t]{0.35\textwidth}{Corresponde al elemento completionThreshold de SCORM} \\ \\  \hline
	parameters & text & Yes & NULL & \parbox[t]{0.35\textwidth}{Corresponde al atributo parameters del elemento item de SCORM} \\ \\  \hline
	isvisible & varchar(5) & Yes & true & \parbox[t]{0.35\textwidth}{Corresponde al atributo isvisible del elemento item de SCORM} \\ \\  \hline
	attemptAbsoluteDurationLimit & varchar(6) & Yes & NULL & \parbox[t]{0.35\textwidth}{Corresponde al atributo attemptAbsoluteDurationLimit del elemento limitConditions de SCORM} \\ \\  \hline
	dataFromLMS & text & Yes & NULL & \parbox[t]{0.35\textwidth}{Corresponde al elemento dataFromLMS del elemento item de SCORM} \\ \\  \hline
	attemptLimit & varchar(10) & Yes & NULL & \parbox[t]{0.35\textwidth}{Número máximo de intentos para la actividad}\\ \hline 
	scormType & varchar(6) & Yes & NULL & \parbox[t]{0.35\textwidth}{Tipo del recurso SCORM}\\ \hline \\
\caption{Estructura de la tabla scos} \label{tab:scorm_scos-structure} \\
\end{longtable}

%
% Structure: scorm_sco_presentations
%
\begin{longtable}{c c c c l}
	\multicolumn{1}{c}{\textbf{Field}} &
	\multicolumn{1}{c}{\textbf{Type}} &
	\multicolumn{1}{c}{\textbf{Null}} &
	\multicolumn{1}{c}{\textbf{Default}} &
	\multicolumn{1}{c}{\textbf{Comments}} \\ \hline \hline
\endfirsthead
	\multicolumn{1}{c}{\textbf{Field}} &
	\multicolumn{1}{c}{\textbf{Type}} &
	\multicolumn{1}{c}{\textbf{Null}} &
	\multicolumn{1}{c}{\textbf{Default}} &
	\multicolumn{1}{c}{\textbf{Comments}} \\ \hline \hline
\endhead \endfoot
	\textbf{\textit{id}} & int(11) & Yes & NULL & \parbox[t]{0.35\textwidth}{Identificador de la presentación de la actividad} \\ \hline 
	hideKey & varchar(10) & Yes & NULL \\ \hline 
	sco\_id & int(11) & Yes & NULL & \parbox[t]{0.35\textwidth}{Identificador del SCO al cuál está relacionada la presentación de la actividad}\\ \hline \\ 
 \caption{Estructura de la tabla sco\_presentations} \label{tab:scorm_sco_presentations-structure} \\
\end{longtable}


\subsection{Tablas relacionadas al plugin Wiki}
%
% Structure: wiki_entries
%
\begin{longtable}{c c c c l}
	\multicolumn{1}{c}{\textbf{Field}} &
	\multicolumn{1}{c}{\textbf{Type}} &
	\multicolumn{1}{c}{\textbf{Null}} &
	\multicolumn{1}{c}{\textbf{Default}} &
	\multicolumn{1}{c}{\textbf{Comments}} \\ \hline \hline
\endfirsthead
	\multicolumn{1}{c}{\textbf{Field}} &
	\multicolumn{1}{c}{\textbf{Type}} &
	\multicolumn{1}{c}{\textbf{Null}} &
	\multicolumn{1}{c}{\textbf{Default}} &
	\multicolumn{1}{c}{\textbf{Comments}} \\ \hline \hline
\endhead \endfoot
	\textbf{\textit{id}} & int(11) & Yes & NULL & \parbox[t]{0.35\textwidth}{Identificador de la página del wiki} \\ \\  \hline
	wiki\_id & int(11) & Yes & NULL & \parbox[t]{0.35\textwidth}{Identificador del wiki al cual está asociada la página} \\ \\  \hline
	member\_id & int(11) & Yes & NULL & \parbox[t]{0.35\textwidth}{Creador de la página} \\ \\  \hline
	title & varchar(255) & Yes & NULL & \parbox[t]{0.35\textwidth}{Título de la página} \\ \\  \hline
	content & text & Yes & NULL & \parbox[t]{0.35\textwidth}{Contenido de la página} \\ \\  \hline
	revision & int(6) & Yes & 1 & \parbox[t]{0.35\textwidth}{Número de revisión de la página} \\ \\  \hline
	created & datetime & Yes & NULL & \parbox[t]{0.35\textwidth}{Fecha de creación de la página} \\ \\  \hline
	updated & datetime & Yes & NULL & \parbox[t]{0.35\textwidth}{Fecha de actualización de la página} \\ \\  \hline
	slug & text & Yes & NULL & \parbox[t]{0.35\textwidth}{Versión para URLs del título de la página} \\ \\
 \caption[Estructura de la tabla entries]{Estructura de la tabla entries. Almacena las entradas (páginas) del wiki} \label{tab:wiki_entries-structure} \\ 
\end{longtable}

%
% Structure: wiki_revisions
%
\begin{longtable}{c c c c l}
	\multicolumn{1}{c}{\textbf{Field}} &
	\multicolumn{1}{c}{\textbf{Type}} &
	\multicolumn{1}{c}{\textbf{Null}} &
	\multicolumn{1}{c}{\textbf{Default}} &
	\multicolumn{1}{c}{\textbf{Comments}} \\ \hline \hline
\endfirsthead
	\multicolumn{1}{c}{\textbf{Field}} &
	\multicolumn{1}{c}{\textbf{Type}} &
	\multicolumn{1}{c}{\textbf{Null}} &
	\multicolumn{1}{c}{\textbf{Default}} &
	\multicolumn{1}{c}{\textbf{Comments}} \\ \hline \hline
\endhead \endfoot
	entry\_id & int(11) & Yes & NULL & \parbox[t]{0.35\textwidth}{Identificador de la página} \\ \\  \hline
	member\_id & int(11) & Yes & NULL & \parbox[t]{0.35\textwidth}{Identificador del miembro que realizó el cambio} \\ \\  \hline
	title & varchar(255) & Yes & NULL & \parbox[t]{0.35\textwidth}{Título de la página} \\ \\  \hline
	content & text & Yes & NULL & \parbox[t]{0.35\textwidth}{Contenido de la revisión} \\ \\  \hline
	revision & int(6) & Yes & NULL & \parbox[t]{0.35\textwidth}{Número de la revision} \\ \\  \hline
	created & datetime & Yes & NULL & \parbox[t]{0.35\textwidth}{Fecha de creación de la revision} \\ \\
 \caption[Estructura de la tabla revisions]{Estructura de la tabla revisions. Almacena las revisiones a las páginas del wiki} \label{tab:wiki_revisions-structure} \\ 
\end{longtable}

%
% Structure: wiki_wikis
%
\begin{longtable}{c c c c l}
	\multicolumn{1}{c}{\textbf{Field}} &
	\multicolumn{1}{c}{\textbf{Type}} &
	\multicolumn{1}{c}{\textbf{Null}} &
	\multicolumn{1}{c}{\textbf{Default}} &
	\multicolumn{1}{c}{\textbf{Comments}} \\ \hline \hline
\endfirsthead
	\multicolumn{1}{c}{\textbf{Field}} &
	\multicolumn{1}{c}{\textbf{Type}} &
	\multicolumn{1}{c}{\textbf{Null}} &
	\multicolumn{1}{c}{\textbf{Default}} &
	\multicolumn{1}{c}{\textbf{Comments}} \\ \hline \hline
\endhead \endfoot
	\textbf{\textit{id}} & int(10) & Yes & NULL & \parbox[t]{0.35\textwidth}{Identificador del wiki}\\ \hline 
	course\_id & int(11) & Yes & NULL & \parbox[t]{0.35\textwidth}{Identificador del curso al cual está asociado el wiki} \\ \\  \hline
	name & varchar(255) & Yes & NULL & \parbox[t]{0.35\textwidth}{Nombre del wiki} \\ \\  \hline
	description & text & Yes & NULL & \parbox[t]{0.35\textwidth}{Descripción del wiki} \\ \\
 \caption[Estructura de la tabla wikis]{Estructura de la tabla wikis. Listado de wikis registrados} \label{tab:wiki_wikis-structure} \\
\end{longtable}
