%!TEX root = ../Libro.tex
\chapter{Características del Sistema}
\section{Justificación de funcionalidades}

\subsection{Núcleo de Ósmosis}
Está formado por los componentes básicos de Ósmosis, sobre los cuáles se construirán e instalarán las herramientas que servirán de soporte a las actividades de los cursos. Dichas herramientas contendrán características que contribuyen a la orientación del curso en cualquiera de los enfoques pedagógicos planteados: constructivismo, cognitivismo, conectivismo o una combinación de ellos.

\subsubsection*{Funcionalidades básicas ofrecidas}
\begin{itemize}
	\item \textbf{Manejo de usuarios}
	\item \textbf{Manejo de cursos}
	\item \textbf{Registro y recuperación de estadísticas}
	\item \textbf{Editor de texto enriquecido con guardado automático}
	\item \textbf{Manejo de palabras claves o etiquetas por usuario}
\end{itemize}

\subsection{Foros}

La herramienta de foros es en sí misma agnóstica en cuanto a la metodología pedagógica que apoya. Pueden ser usados desde un enfoque eminentemente cognitivista, por ejemplo, una sesión de preguntas y respuestas sobre un tema, entre el profesor y el estudiante, puede tener como medio de comunicación e intercambio de ideas un foro. Desde este enfoque hay un máximo control sobre lo que los alumnos pueden hacer y la forma en que pueden participar de la actividad.

Así mismo, los foros pueden ser utilizados bajo la óptica constructivista, según la cual el docente puede indicarle a sus alumnos que discutan libremente sobre un tema o bajo lineamientos. Así mismo el docente actúa como un moderador de la actividad y es quien establecerá las conclusiones finales a partir de lo que se haya expuesto.

Una nueva visión puede ayudar a entender lo que sucede actualmente con los foros en la red, la teoría del conectivismo. Con esta herramienta el comportamiento de sus participantes puede parecer caótico, pero usualmente se encamina por medio de la interacción de los que allí escriben. Esto permite una libertad mucho mayor que la que se logra con el constructivismo, lo cual fomenta la creatividad y el flujo de la información.

\subsection*{Justificación de funcionalidades propias}

\begin{itemize}
\item \textbf{Granularizar permisos de acceso:} Apoyado por el cognitivismo y el constructivismo puesto que al limitar o ampliar la capacidad de participación sobre un elemento del curso, el docente tendrá la capacidad de guiar el mismo y enfocar su enseñanza en actividades con significado. Además permite la creación foros pro subgrupos dentro de la clase y el manejo de los mismos por figuras como las de un ``preparador''.
\item \textbf{RSS de los posts:} Se apoya en uno de los principios del conectivismo que es la de mantener la información actualizada e interconectada entre los nodos que la generan. Proveer una manera de sindicar lo que se escribe en los foros permite mantener actualizado al estudiante y al docente sin que ellos estén obligados a visitar la aplicación para leer los mensajes que le interesen. Simplemente necesitaría su lector preferido de RSS.
\item \textbf{Puntuaciones a los posts:} Se apoya en el principio del conectivismo que afirma que la intención del proceso es la información precisa. Una manera de asignar relevancia a una información dentro de una red es dándole una puntuación de valor. Las entradas del foro que tengan más relevancia con el tema o aporten algo realmente valioso serán mejor puntuadas por los participantes del mismo. Se puede proveer un mecanismo mediante el cual se diferencie o se le asigne \textbf{más peso a la puntuación} de un profesor que a la de un estudiante
\item \textbf{Tipos de Foro:} Los foros, según el alcance que tengan pueden ser privados para los participantes de un curso, públicos, compartidos entre varios cursos o privados para un grupo dentro de un curso. Los foros privados sirven como herramienta de comunicación alumno-profesor bajo una óptica cognitiva/constructiva, mientras que los foros privados entre grupos tienen como finalidad la comunicación entre los estudiantes y favorece las actividades en las que la creatividad o trabajo propio del estudiante es requerido. Los foros de tipo pregunta-respuesta caen también en la categoría del constructivismo, dado que es necesaria la creatividad del estudiante para responder la pregunta o aportar soluciones a un problema. No obstante, los foros públicos y compartidos entre curso también pueden ser enfocados bajo metodologías pedagógicas tradicionales, facilitan el intercambio de la información, la apertura de la misma además de la libertad de creación, valores todos asociados con el conectivismo.
\end{itemize}

\subsection{Intercambio de Archivos}

El intercambio de archivos es una herramienta que favorece cualquier enfoque pedagógico. En el caso cognitivista puro, sirve para la entrega de material de apoyo y estudio por parte del instructor al alumno, así como para la entrega de asignaciones. Desde el enfoque constructivista y constructivista social, las razones para el intercambio de archivos no son muy distintas a las usadas en la visión anterior. No obstante vista desde las perspectiva conectivista, encontramos que el intercambio de archivos favorece en buena medida el intercambio entre los nodos de la red de aprendizaje y aumenta la cantidad de conocimiento que ya se posee. 

El ejemplo más típico de como el intercambio de archivos puede favorecer la interconexión entre los nodos de la red se da cuando un alumno publica una guía de estudio o un conjunto de ejercicios resueltos para un curso, estos se hacen inmediatamente disponibles para todos en la red, de modo que eventualmente alguien más, que los encuentre útiles, los reseñará y utilizará para su propio estudio. Finalmente, otra persona en busca de algo parecido encontrará la dicha reseña y entenderá que si para él fueron útiles, probablemente valga la pena utilizarlos y reseñarlos igualmente.


\subsection*{Justificación de funcionalidades propias}

La idea detrás del intercambio de archivos dentro de Ósmosis2, es diferente a la acostumbrada en un correo electrónico, que es la de adjuntar archivos a un mensaje dirigido a una o más personas. En lugar de enviar archivos, se trata de que las personas ``recojan'' los mismos en el ``casillero'' de quién los generó.

En el caso de entregar una asignación a un profesor, la metáfora más correcta no es que el profesor la busque en los casilleros de los alumnos, sino que los alumnos la entregue en el casillero del instructor. En este caso si podemos hablar de una entrega de archivos.

\begin{itemize}

\item \textbf{Casillero personal publico:} Permite al dueño del casillero dejar archivos para que sea vistos por cualquier persona en la red.
\item \textbf{Casillero de entregas para el profesor:} Permite a los integrantes de un curso entregar asignaciones o trabajos al instructor del mismo, siendo visibles estos archivos solamente por el profesor luego de su entrega.
\item \textbf{Casillero personal privado:} Aunque parezca deseable el tener un casillero privado, se debe entender que una plataforma de aprendizaje concebida con una filosofía centrada en el intercambio, desde sus inicios, no es compatible con esta idea. Uno de los usos más comunes que se le ha dado a esta funcionalidad es la de ocultar contenidos de materias que no están siendo dictadas, pero que lo estarán nuevamente en un futuro. No obstante, este uso, visto desde la perspectiva conectivista, significa que se está sustrayendo conocimiento de la red, o lo que es lo mismo, cuando una persona quiera hacer referencia a un archivo que consideraba existente, encontrará que el mismo ya no está disponible.
\item \textbf{Descripción de archivos y comentarios:} El usuario tendrá la posibilidad de agregar una descripción o comentario al archivo que coloca en cualquiera de los casilleros donde tenga permiso, y también habrá la posibilidad de que el resto de la comunidad (según sea el permiso de lectura del archivo) pueda comentar sobre la calidad o contenido del mismo.
\end{itemize}

\subsection{Mensajería Interna}
La mensajería interna permitirá recibir los mensajes a aquellos usuarios más sensibles con su privacidad o que no desean recibir correo electrónico de la plataforma. La existencia de una mensajería interna fomenta, además, un uso más frecuente del sistema, y por ende mayor participación de los usuarios en los cursos que integran.


\subsection*{Justificación de funcionalidades propias}

\begin{itemize}

\item \textbf{Envío de mensajes a correo externo:} Para aquellos que no deseen revelar su dirección de correo, pero aún así prefieren leer sus mensajes desde su buzón preferido, se les da la oportunidad de reenviar automáticamente los mensajes que reciban en su correo interno directamente a una dirección de correo electrónico externa.
\item \textbf{Libreta de direcciones en microformato hCard:} El uso de microformatos para el manejo de conocidos ayudará al explorador a reconocer la herramienta favorita del usuario para comunicarse con ellos o ubicarlos en el mapa.

\end{itemize}


\subsection{Blogs}

Los Blogs son uno de los pilares fundamentales de Ósmosis2. Esta herramienta permite que los usuarios expresen sus opiniones sobre los cursos, expongan sus ideas y muestren al resto de la comunidad cualquier tema que pueda ser de interés general. Igual que otras funcionalidades, esta no está atada a ningún enfoque pedagógico en especial, no obstante poco tiene que ver con el cognitivismo puro. 
Un blog puede ser utilizado por el docente como una herramienta constructivista, mediante la cual puede comentar y dirigir las ideas de sus estudiantes con respecto a un tema específico. Bajo el enfoque conectivista, el blog, es una fuente generadora de nodos de información, que al interconectarse con otros nodos, genera conocimiento.

La idea general para los blogs dentro de Ósmosis 2 es que cada usuario tenga su blog personal. Por lo tanto, no existe un blog para cada curso, puesto que este tipo de herramienta puede considerarse, realmente, personal y no sería útil bajo un esquema dentro de una asignatura.

Este esquema favorece la interconexión de la información y la libertad de creación, que son principios del conectivismo y el constructivismo.

\subsection*{Justificación de funcionalidades propias}

\begin{itemize}


\item  \textbf{Agregador de feeds RSS/Atom externos:} Permite que aquellos usuarios que ya posean un blog y no deseen usar el suministrado por la plataforma Ósmosis2, puedan seguir aportando ideas de la misma manera que lo harían aquellos que sí utilizan la herramienta interna, mediante la agregación de los RSS que genere su aplicación.

\end{itemize}

\subsection{Wiki}

Al igual que los Blogs, el wiki es una herramienta de gran utilidad para los enfoques constructivista y conectivista. Los wikis favorecen el principio del conectivismo que afirma que la capacidad de generar conocimiento es más importante que el conocimiento que ya se posee.

En Ósmosis2, y a diferencia de los Blogs, los wikis son una posesión grupal y no individual. Es por esto que cada curso tendrá la posibilidad de tener un Wiki propio, cuyos contenidos se irán agregando automáticamente al Wiki global de la aplicación.


\subsection*{Justificación de funcionalidades propias}

\begin{itemize}

\item \textbf{Alcance del Wiki:} El wiki pueden ser configurado por el administrador de modo que sean accesibles sólo para los participantes del curso; en lugar del la configuración predeterminada que es la de wikis púbicos.

\end{itemize}

\subsection{Chat}
La principal función del chat es contribuir a la aplicación de los enfoques pedagógicos conectivista y cognitivista. El enfoque cognitivista se ve reflejado en las actividades que se lleven a cabo con la participación del profesor, como impartir lecciones, fomentar sesiones de preguntas y respuestas, entre otras. El enfoque conectivista se podría evidenciar en las discusiones o conversaciones entre alumnos, en tiempo real, en las cuales se compartan ideas y conocimientos.

\subsection*{Justificación de funcionalidades propias}

\begin{itemize}

\item \textbf{Suspensión (por parte del profesor) de estudiantes:} Permite a los profesores suprimir, por un período definido o permanentemente, el derecho de participación de los estudiantes que hayan incumplido alguna de las normas establecidas para el chat o cualquier otra funcionalidad de interés.

\item \textbf{Historial de conversaciones:} Ofrece a los usuarios del chat la posibilidad de guardar las conversaciones en las cuales han participado. De esta forma es posible conservar datos, preguntas o respuestas que sean consideradas útiles para el estudio o la evaluación del contenido del curso.

\end{itemize}

\subsection{Pizarra}
La pizarra es, al igual que el chat una herramienta que favorece el enfoque cognitivo del aprendizaje. El uso que se le puede dar no es muy diferente a lo que sucede en un aula de clase tradicional. El profesor se dirige a sus estudiantes mediante el uso de elementos audiovisuales y ellos atienden a la clase participando eventualmente.

\subsection{Lecciones}

Las lecciones favorecen la implantación de cursos a distancia en los cuales el facilitador o docente puede controlar completamente la forma en que debe aprender el estudiante, lo cual es un enfoque plenamente cognitivista. Con el uso de esta herramienta es posible que el profesor indique que recursos debe utilizar el alumno y en qué orden, además de tener la posibilidad de medir progresivamente el avance del mismo a través uso de pruebas que son requisitos para continuar a la siguiente lección.

\subsection{Enlaces}

Los enlaces sirven como herramienta de los enfoques pedagógicos cognitivista, ya que a partir de ellos se podrá extraer la información que permitirá generar el conocimiento sobre uno o varios contenidos del curso, y conectivista porque es posible compartirlos con otros usuarios o hacerlos públicos de manera tal que todos tengan acceso a la información.

\subsection*{Justificación de funcionalidades propias}

\begin{itemize}

\item \textbf{Guardar enlaces privados/públicos:} Los usuarios podrán guardar la ruta o URL de aquellos sitios que consideren de interés. Los enlaces pueden ser guardados como privados cuando se desea cierta confidencialidad de la información que contiene el enlace mostrado. Pueden ser públicos cuando se quiere que todos los usuarios utilicen dichos enlaces como recurso común para la extracción de información.\\
\textbf{Nota}: Los enlaces privados están representados por las referencias utilizadas por los profesores o las empleadas por los estudiantes en sus grupos de estudio o trabajo

\end{itemize}
\subsection{Calendario/Agenda}

La función principal tanto del calendario del curso como de la agenda es permitir la planificación y desarrollo, de forma ordenada, de las actividades.\\

El calendario y la agenda presentan la misma funcionalidad, su diferencia radica en el tipo de usuario al cual están destinados. El calendario está diseñado, principalmente, para el profesor mientras que la agenda está destinada tanto a profesores como a estudiantes, siendo estos últimos los usuarios frecuentes.

\subsection*{Justificación de funcionalidades propias}

\begin{itemize}

\item \textbf{Profesores pueden guardar eventos en el calendario:} Los profesores podrán colocar en el calendario las actividades programadas para el período de clases en curso, de esta forma se facilita el seguimiento de las actividades pautadas.

\item \textbf{Estudiantes tienen calendario privado para programación personal. Dicha programación es compartible:} Los estudiantes pueden crear un calendario personal para la planificación de sus actividades. En caso de realizar planificaciones para trabajos en grupo se tiene la posibilidad de compartir el calendario con el resto de los integrantes del mismo.

\item \textbf{Los eventos se pueden enlazar a archivos, actividades, enlaces, etc. del curso:} Los usuarios pueden relacionar los eventos plasmados en el calendario o la agenda con archivos que describan de forma detallada la actividad, enlaces que faciliten la búsqueda de información o cualquier otro recurso que permita llevar a cabo la tarea indicada.

\item \textbf{Publicación de eventos usando microformatos (hCalendar) y XML, iCal:} Estas opciones permiten que el usuario que así lo prefiera use la aplicación qué más le agrade para visualizar los eventos del calendario, como pueden ser Google Calendar, iCal, entre otros.

\end{itemize}

\subsection{Capacidades de búsqueda}

La capacidad de búsqueda es una ayuda que se les brinda a los usuarios que no pueden ubicar de una manera rápida las funcionalidades o recursos que hay dentro de un curso. Se espera que cada módulo implemente capacidades de búsqueda sobre los elementos que sea relevante recuperar para llevar a cabo la actividad deseada.

\subsection{Agregador de Noticias}

Su función es aprovechar los contenidos publicados por otros sitios de interés y publicar, dentro del sistema, dichas noticias o artículos.

\subsection{Grupos de Trabajo}

Esta herramienta apoya los enfoques pedagógicos del constructivismo y el constructivismo social, ya que genera el intercambio de información y conocimiento entre los miembros de un grupo. Los participantes actúan como fuente de conocimientos y entre todos se crea una red que contribuye a la adquisición de nuevos conceptos o ideas.

\subsection*{Justificación de funcionalidades propias}

\begin{itemize}

\item \textbf{Asignación de usuarios a grupos de trabajo (Manual y aleatoria):} los integrantes de los grupos de trabajo pueden ser asignados directamente por el profesor o el sistema puede generar un grupo de trabajo a partir de la lista de estudiantes inscritos en el curso.

\item \textbf{Los módulos manejan la noción de grupos de trabajo y clases enteras:} Cada módulo posee herramientas que perimiten realizar las actividades del curso, ya sea a través de grupos de trabajo o de forma individual para todos los estudiantes de una clase. De igual manera se presentan funcionalidades específicas para  actividades en grupo.

\end{itemize}

\subsection{Ayuda del Sistema}
Poseer un entorno uniforme de ayuda para el sistema y sus módulos mejorará la experiencia de los usuarios.

\subsection*{Justificación de funcionalidades propias}

\begin{itemize}

\item \textbf{Tour virtual de cada módulo:} Proporciona a los estudiantes y profesores información general sobre el funcionamiento y las características de cada módulo del sistema.

\item \textbf{Ayuda contextual dentro de cada módulo:} Contiene información específica dentro del contexto de  cada herramienta de manera tal que el usuario pueda obtener la ayuda necesaria fácilmente y sin necesidad de buscar en toda la documentación del usuario.

\end{itemize}

\subsection{Comunidad}

La creación de una comunidad alrededor de temas educativos es importante para mantener la vigencia y frescura de los contenidos. De esta manera se apoya no solo el conectivismo sino que la disponibilidad de contenidos diversos puede ayudar a los profesores que decidan hacer uso de una metodología cognitivista, o a los estudiantes que trabajan en una actividad constructivista.  

\subsection*{Justificación de funcionalidades propias}

\begin{itemize}

\item \textbf{Los usuarios se pueden registrar en los cursos que deseen:} Permite a los estudiantes inscribirse, como oyentes, en cursos de su interés.

\end{itemize}

\subsection{Portafolios} 

Permite crear un repositorio digital que mantenga un registro de las actividades realizadas, los títulos académicos, premios o certificaciones obtenidas, referencias personales, entre otros componentes que proporcionen información sobre las competencias adquiridas por el usuario. 

\subsection*{Justificación de funcionalidades propias}

\begin{itemize}

\item \textbf{El programa de los cursos completados y aprobados se agregan automáticamente al portafolio personal:} Una vez finalizado y aprobado un curso, el sistema lo agrega al historial de actividades realizadas por el usuario. De esta manera quedan reflejados los conocimientos adquiridos en dicho curso.

\item \textbf{Publicación del portafolios con microformatos (hResume):} Brinda la posibilidad de utilizar HTML semántico para exportar los datos y el portafolios.


\item \textbf{El portafolios se puede exportar para ser impreso o enviado digitalmente:} Le brinda al propietario del portafolio la posibilidad de convertir el currículum virtual a un formato que le facilite la impresión, para entregas en físico, o el envío por correo electrónico.

\end{itemize}

\subsection{Registro e integración}

\begin{itemize}
\item \textbf{Api para obtener información de los usuarios de un curso:} Esta funcionalidad permitiría a otras instancias de la institución educativa a utilizar la aplicación como una base de datos para su utilización en otros sistemas.
\end{itemize}

\subsection{Administración automatizada de pruebas}

Las pruebas son herramientas disponibles para el profesor, cuya finalidad es medir el desempeño del estudiante en el curso. El profesor puede aplicarlas siguiendo el enfoque pedagógico de su preferencia. Algunos de los tipos de prueba son:
\begin{itemize}
\item Selección múltiple
\item Verdadero- falso
\item Matching
\item Ordenamiento
\item Preguntas abiertas
\item Preguntas cerradas
\item Completar el espacio en blanco
\item Encuestas
\item Ensayos
\item Preguntas que contengan imágenes, video o audio
\item Definidas por el usuario
\end{itemize}


\subsection*{Justificación de funcionalidades propias}

\begin{itemize}

\item \textbf{Generar preguntas y respuestas de forma aleatoria:} Brinda la posibilidad de que el sistema seleccione, dadas varias opciones de preguntas, aquellas que serán colocadas en la prueba.
\item \textbf{El sistema debe soportar un editor que permita la inclusión de fórmulas:} El sistema suministra un editor de texto que permita construir fórmulas para aquellas preguntas o respuestas que así lo requieran. 
\item \textbf{Los profesores pueden establecer el tiempo límite de las pruebas:} Las pruebas presentadas de forma virtual pueden tener, al igual que las pruebas presenciales, un límite de tiempo en el cual deben ser completadas. El profesor puede establecer el tiempo estimado para la realización de las pruebas.
\item \textbf{Los profesores pueden mostrar las respuestas correctas como feedback:} Una vez que las pruebas hayan sido completadas por todos los estudiantes el profesor puede publicar las respuestas correctas de las preguntas contenidas en la prueba.
\item \textbf{El sistema debe proveer seguridad sobre las pruebas:} El sistema debe garantizar que cada prueba pertenezca y pueda ser modificada sólo por un estudiante y por el profesor encargado del curso.
\item \textbf{Pruebas restringidas por dirección IP:} Cada prueba será asociada con una dirección IP, como una de los mecanismos de seguridad aplicados a las pruebas.

\end{itemize}

\subsection{Manejo de Puntuaciones}

\subsection*{Justificación de funcionalidades propias}

\begin{itemize}

\item \textbf{Se puede puntuar cada estudiante en todas las preguntas o cada pregunta de todos los estudiantes:} El profesor tiene la libertad de corregir toda la prueba, para cada estudiante, o corregir las pruebas de todo el curso por pregunta (para todos los estudiantes se corrige la misma pregunta y una vez finalizada la corrección de ésta se avanza a la siguiente. Este proceso se repite hasta completar la corrección de toda la prueba)

\end{itemize}

\subsection{Control de calificaciones del curso (gradebooks)}

\subsection*{Justificación de funcionalidades propias}

\begin{itemize}

\item \textbf{El profesor puede agregar calificaciones para actividades off-line:} Permite incorporar a la tabla de calificaciones aquellas actividades que no son realizadas de forma virtual y que deben ser evaluadas.

\item \textbf{Se pueden exportar las calificaciones contenidas en la tabla:}Brinda la posibilidad de convertir el tabla de calificaciones presentada en formato digital a un formato que le facilite la impresión o el envío por correo electrónico.

\item \textbf{Se puede crear una escala de evaluación que contemple porcentajes, status de aprobado/reprobado, calificación alfabética:} El profesor o preparador puede establecer la escala de puntuación de su preferencia o que se adapte a sus necesidades. 

\end{itemize}

\subsection{Seguimiento de usuarios}
Esta herramienta permite medir el desempeño o el progreso del estudiante bajo cualquiera de los enfoques pedagógicos propuestos.


\subsection*{Justificación de funcionalidades propias}

\begin{itemize}

\item \textbf{Estadísticas de lo que más hace un estudiante/ todos los estudiantes:} Permite llevar un registro de aquellas actividades o herramientas que son más frecuentadas dentro de un curso. Esto podría suministrar la información para que el profesor, si así lo desea, tome en cuenta como evaluación la participación de cada estudiante dentro del sistema.

\item \textbf{Historial de navegación de cada estudiante:} Ofrece la posibilidad de mantener un informe detallado de las herramientas del sistema utilizadas por los estudiantes inscritos en un curso.

\end{itemize}

\subsection{Accesibilidad}


\subsection*{Justificación de funcionalidades propias}

\begin{itemize}

\item \textbf{La aplicación entera debe satisfacer los estándares de accesibilidad, tanto para la visualización como para la edición de cursos:} 
\item \textbf{La edición de los campos de texto debe tener un checker de accesibilidad:} La implementación de dicha revisión debe hacerse de tal manera que cualquier usuario pueda corregir los errores.

\end{itemize}

\subsection{Compartir/Reusar}

El sistema brinda la posibilidad de compartir la información, de acuerdo a la relevancia o potencial utilidad que tenga para los estudiantes. Esto puede o no contribuir con el enfoque pedagógico conectivista de acuerdo a la apertura que le de el autor al material pedagógico.


\subsection*{Justificación de funcionalidades propias}

\begin{itemize}

\item \textbf{Repositorio de learning objects:} la existencia de un repositorio de learning objects le permite a los profesores guardar y, opcionalmente, publicar de manera estructurada uno o varios recursos pedagógicos así como encontrar otros recursos que puedan ser de su interés.
\item \textbf{Conexión a repositorios externos el acceso a otros repositorios:} permitirá una mayor diversidad en los enfoques y en el nivel de calidad de los recursos.
\item \textbf{Palabras claves o etiquetas para los learning objects:} De manera de facilitar la búsqueda de los recursos almacenados por el sistema, se permitirá a los usuarios crear una glosario por medio del etiquetado de los recursos.
\item \textbf{Capacidad de importar y exportar el curso en SCORM, IMS, AICC:} Provee al profesor la oportunidad de exportar el contenido de un curso de manera que pueda ser utilizado por los estudiantes o aquellas personas interesadas, aún si no están conectados a internet. 

\end{itemize}

\subsection{Otras funcionalidades} 

\begin{itemize}
 
\item \textbf{Los cursos pueden ser cerrador o eliminados:} Los cursos pueden ser cerrados, una vez finalizado el período de clases en curso, o eliminados si el curso no está considerado para ser abierto nuevamente.

\item \textbf{Los contenidos de los cursos eliminados no se pueden recuperar:} Si el curso es eliminado todas la herramientas y contenidos relacionados con él son eliminados. Por esta razón no existe la posibilidad de recuperarlos.

\item \textbf{Los contenidos de los cursos cerrados se restauran una vez que el curso se reabre:} Si el curso es cerrado su contenido y herramientas permanecen  inalteradas hasta el momento en que se reabra el mismo.

\item \textbf{Al reabrir un curso, se da la opción para restaurar los profesores y estudiantes o no:} al momento de reabrir el curso se permitirá seleccionar a los usuarios que se desea mantener inscritos en el curso. 

\item \textbf{Utilizar learning objects para el almacenamiento de pruebas anteriores o modelos de prueba que puedan ser utilizados como material de estudio} El profesor, al registrar una prueba podrá almacenarla como learning object. Opcionalmente se podrá almacenar junto a las respuestas correctas.  
    
\item \textbf{Exportar a:} El sistema y sus módulos permitirán exportar en PDF o imprimir los contenidos.

\item \textbf{Opción para la herramienta de manejo de archivos:} manejo de versiones para archivos, detección de virus al subir y descargar.

\end{itemize}