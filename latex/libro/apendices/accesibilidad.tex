%!TEX root = ../Libro.tex
\section{Pautas de Accesibilidad Web}
\label{apendice_accesibilidad}
Las pautas dictadas por la W3C \citep{AccesabilidadESC2007} contienen información técnica que se ha intentado resumir y explicar en a continuación:
\begin{enumerate}
	\item \textbf{Proporcione alternativas equivalentes para el contenido visual y auditivo}\\  Con la intención de permitir el acceso a contenidos no textuales a personas discapacitadas.
	%Esto parece ser una explicación a una problemàtica del sistema actual, no considero que vaya en el marco teórico.%
	%El problema, dentro del sistema que se plantea, es que los contenidos son generados por muchos usuarios por lo que a falta de una normativa institucional y una cultura sobre accesibilidad en la Universidad Simón Bolívar, el sistema optará por hacer recomendaciones puntuales en cada herramienta.\\ **Por ejemplo:** al agregar un nuevo vídeo, imagen u otro, se le presentará al usuario un cuadro para escribir una descripción del contenido del medio. Sin embargo no será obligatorio.%
	\item \textbf{No se base sólo en el color}\\ Los colores no deben ser usados como guias visuales, o al menos deben proveerse alternativas de tal modo que las personas con discapacidades visuales reciban la misma información. Se debe asegurar el contraste entre el fondo y el primer plano en el texto y en las imágenes.
	\item \textbf{Utilice marcadores y hojas de estilo y hágalo apropiadamente.}\\ El uso de las etiquetas de marcaje definidas en html deben ser usadas de acuerdo a su semántica asociada. Se destacan los siguientes usos inapropiados:
	\begin{itemize}
	\item El uso de la etiqueta BLOCKQUOTE para generar sangrías cuando el uso adecuado es para citar textos.
	\item El uso de etiquetas de títulos (H1, H2, etc.) para manipular el tamaño de la letra.
	\end{itemize}
	\item \textbf{Identifique el idioma usado}\\ Identificar el idioma predominante del contenido y de las distintas secciones (en caso de que cambie) permite a los navegadores especializados cambiar el idioma y adecuarse para reconocer el cambio de idioma. Así mismo se debe expandir el significado de los acrónimos (ACRONYM) y abreviaciones (ABBR) la primera vez que aparecen en el documento.
	\item \textbf{Cree tablas que se transformen correctamente}\\ Las tablas deben ser usadas únicamente para información tabular (tablas de datos) y no como elementos para hacer la maquetación de los contenidos. Se debe hacer uso del marcaje para definir los encabezados, resumen, agrupamiento de celdas y demás dentro de la tabla.
	\item \textbf{Asegúrese de que las páginas que incorporan nuevas tecnologías se transformen correctamente.}\\ Los contenidos deben ser legibles y comprensibles cuando las hojas de estilos son desactivadas, cuando las capacidades de scripting del lado del usuario (Javascript) estén deshabilitadas, y que los contenidos accesibles por medios dinámicos sean accesibles sin dichas capacidades. Debe hacerse uso de Javascript no obstructivo \citep{Accesabilidad_JS2007}.
	\item \textbf{Asegure al usuario el control sobre los cambios de los contenidos tempo-dependientes}\\ Debe permitirse a los usuarios detener cualquier contenido que presente algún tipo de movimiento o parpadeo.
	\item \textbf{Asegure la accesibilidad directa de las interfaces de usuario incrustadas}\\ Los objetos incrustados (applets y otros) deben asegurar los principios de un diseño accesible: funcionalidad de acceso independiente del dispositivo, teclado operable, voz automática, etc.
	\item \textbf{Diseñe para la independencia del dispositivo}\\ Permita la activación de los elementos independientemente de los dispositivos de entrada. Definir el orden en que se recorren los elementos de la interfaz con el tabulador (tabindex) y métodos abreviados mejoran la accesibilidad (sin embargo, estos últimos deberán ser definidos por el usuario a fin de evitar conflictos con métodos abreviados del sistema operativo).
	\item \textbf{Utilice soluciones provisionales}\\  Utilice soluciones de accesibilidad provisionales de forma que las ayudas técnicas y los antiguos navegadores operen correctamente. (Estas indicaciones perdieron vigencia debido a los avances desde la creación del documento - en 1999)
	\item \textbf{Utilice las tecnologías y pautas W3C}\\ El uso de tecnologías no desarrolladas por la W3C como shockwave y PDF deben ser vistos como plugins y evitados en todo lo posible debido a que no ofrecen las características accesibles ofrecidas por la tecnología de la W3C. En la realidad, los avances en accesibilidad se pueden constatar en la publicación las guias de accesibilidad para PDF y para Flash publicadas por Adobe en \url{http://www.adobe.com/accessibility/}. Sin embargo, se recomienda disponer una página accesible como alternativa.
	\item \textbf{Proporcione información de contexto y orientación}\\ Algunos elementos como listas de items pueden ser difíciles de manejar así como las etiquetas asociadas a los elementos de un formulario, se recomienda separarlos por medio del uso de OPTGROUP en el primer caso y usar la etiqueta LABEL en el segundo.
	\item \textbf{Proporcione mecanismos claros de navegación}\\ Se debe identificar el ``destino'' o funcionalidad de un enlace, agregar metadatos, tablas de contenidos o mapas del sitio, agrupar los vínculos relacionados (barras de navegación), si se proporcionas funcionalidades de búsqueda se deben permitir distintos niveles de habilidad, localizar la información relevante al principio del párrafo y hacer los títulos relevantes y entendibles fuera de contexto (ayuda al hojeo del contenido), si un documento se extiende por varias página proporcione dicha información con la etiqueta LINK y sus atributos rel y rev (\url{http://www.seoconsultants.com/meta-tags/link-relationship.asp}) y permita maneras de saltar elementos o grupos de elementos que sean comúnmente ignorados cuando se busca leer el contenido (navegación, búsqueda, etc.)
	\item \textbf{Asegúrese de que los documentos sean claros y simples}\\ Se debe asegurar que las imágenes tienen textos equivalentes (atributo alt) de modo que los discapacitados visuales, o quienes no pueden o han elegido no ver los gráficos tengan acceso a la información. Así mismo, la utilización de un lenguaje claro y simple promueve una comunicación efectiva.
\end{enumerate}