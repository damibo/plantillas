%!TEX root = Libro.tex
\section*{Introducción}
\addcontentsline{toc}{section}{Introducción}
Con la masificación del acceso a la tecnología, y en especial a Internet, la labor pedagógica se ha visto en la obligación de ponerse al día con las nuevas herramientas existentes, no solo con la intención de garantizar una máxima asimilación de los contenidos, sino también para extender la labor educativa a personas que, anteriormente, no tenían acceso al conocimiento.\\

La Universidad Simón Bolívar no ha sido la excepción; aprovechando el acceso masivo a Internet como herramienta viable y de bajo costo para implantar una plataforma que sirva como complemento para los cursos que se dictan en sus aulas. En la actualidad, la institución cuenta con un portal educativo basado en un ambiente Web, la cual brinda apoyo al docente en la entrega de material complementario y acercamiento a los alumnos de los cursos que dicta.\\

Este es el camino que han tomado, con mayor o menor éxito, muchas otras entidades educativas alrededor del mundo. El software que apoya estas plataformas se ha ido desarrollando sobre la marcha, conforme las necesidades hayan ido demandando nueva funcionalidades, y los avances tecnológicos así lo hayan permitido. Pero en gran parte han sido desarrollados sin una concepción previa de cómo deben ser utilizados estos nuevos recursos para sacar el mayor provecho pedagógico a los mismos.\\

En la actualidad el área de mayor auge en Internet son las comunidades, en donde grupos de usuarios que se dedican a compartir contenidos multimedia los cuales pueden comentar, calificar y clasificar; fenómeno que algunos han llamado ``Web 2.0''. Este modelo de interacción, en el cual el usuario es generador de contenidos, se ha probado a si mismo y a sus usuarios como muy útil, puesto que se aprovecha al máximo la capacidad de comunicación, casi instantánea, que brinda la tecnología para la generación de información valiosa a la que todos pueden acceder.\\

Surge entonces la siguiente pregunta, ¿por qué no usar este modelo para facilitar la labor educativa? Parte de la respuesta contempla el evaluar el uso que se le está dando a las plataformas que ya se poseen e intentar adaptarlas a este nuevo enfoque.\\

El presente proyecto nace de dicha necesidad de replantear las herramientas utilizadas actualmente por la Universidad Simón Bolívar, de manera que estas se transformen en facilitadores al servicio de quienes van a utilizarlas y se adecuen lo mejor posible a las estructuras que se manejan en el ambiente educativo, abriendo paso, a su vez, al nuevo paradigma de la educación a distancia.\\

Para ello se requiere evaluar y seleccionar aquellos componentes que mejor se adapten al nuevo ambiente de aprendizaje y que a su vez estimulen no sólo la adquisición de conocimientos, sino la generación de los mismos. Con esto, es posible plantear un nuevo enfoque que supone hacer más énfasis en la comunidad de usuarios (estudiantes y profesores) que en los contenidos, sin descuidar la calidad de los últimos y que, además, busca reflejar, virtualmente, la interacción física que se observa en las aulas de clase.\\