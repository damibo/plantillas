%!TEX root = Libro.tex
\begin{large}
	\begin{center}
		\textbf{\titulo}
	\end{center}
\end{large}
\begin{large}
	\begin{center}
		Por \\
		\autores
	\end{center}
\end{large}

\begin{center}
	\textbf{RESUMEN}
\end{center}

Con la masificación del acceso a la tecnología, y en especial a Internet, la labor pedagógica se ha visto en la obligación de ponerse al día con las nuevas herramientas existentes, no solo con la intención de garantizar una máxima asimilación de los contenidos, sino también para extender la labor educativa a personas que, anteriormente, no tenían acceso al conocimiento.\\

La nueva tendencia en el uso de Internet, denominada Web2.0, apuntan a que los sitios más exitosos serán aquellos que logren aprovechar la inteligencia colectiva para generar contenidos con valor agregado. Pero para alcanzar ese conocimiento colectivo es necesario alcanzar masa crítica, para lograrlo es necesario ofrecer un servicio de alta calidad y en constante evolución.\\

Implementar esta nueva tendencia en un entorno de aprendizaje en línea es el principal objetivo del presente proyecto de grado. Partiendo de la plataforma existente en la Universidad Simón Bolívar, se plantea desarrollar su reestructuración de modo que se convierta en un repositorio navegable del conocimiento organizacional generado dentro del campus.\\

La meta de este trabajo fue la implementación de un Sistema de Manejo del Aprendizaje (LMS, por sus siglas en inglés), que ofrezca a la comunidad universitaria un entorno educativo con las características propias de la Web2.0 y a sus administradores un entorno de fácil mantenimiento que permita el rápido crecimiento y evolución por medio del desarrollo de nuevas funcionalidades.\\

Ósmosis2, como se denomina este proyecto, ofrecerá a los estudiantes y profesores una nueva forma de aprender y de dejar un legado a toda la comunidad.
