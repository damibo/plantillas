\documentclass{article}

\renewcommand{\abstractname}{Resumen}

\usepackage{ucs}
\usepackage[utf8x]{inputenc}
\usepackage{fancyhdr}
\usepackage{hyperref}
\pagestyle{fancy}
\usepackage{palatino}


% cabecera
\lhead{
	\footnotesize{
		\textbf{GNU/Ósmosis 2.0} \\
		\textbf{Documento de Visión}}
}
\chead{}
\rhead{
	\footnotesize{
		\textbf{Versión:} \textnormal{$<$0.1$>$}\\
		\textbf{Fecha:} \textnormal{$<$Agosto 21 de 2007$>$}}}
\lfoot{}
\rfoot{}
\cfoot{\thepage}


\usepackage[top=3cm, bottom=3cm, left=3cm, right=3cm]{geometry}

%\usepackage{times}
\makeindex
\begin{document}

% Article top matter
\title{GNU/Ósmosis 2.0 \\ Visión}
\author{Joaquín Windmuller \texttt{02-35564} \\
		Universidad Simón Bolívar}
\date{\today}  %\today is replaced with the current date
\maketitle

\begin{abstract}
	Resumen
\end{abstract}

\pagebreak
\tableofcontents
\pagebreak

\section{Introducción}
	\subsection{Propósito}
	El propósito de este documento es recolectar, analizar y definir las necesidades y requerimientos del Sistema GNU/Ósmosis 2.0. Se enfoca en las funcionalidades requeridas por los stakeholders y los usuarios finales, y las razones de su existencia. Los detalles de cómo el sistema satisface estos requerimientos se encuentran plasmados en los casos de uso y especificaciones que complementan este documento.

	\subsection{Definiciones, Acrónimos y Abreviaciones}	
		\begin{itemize}
			\item \textbf{USB:} Universidad Simón Bolívar
			\item \textbf{Sistema de Aulas Virtuales de la USB:} sistema de educación a distancia disponible en la USB desarrollado sobre el Sistema GNU/Ósmosis 1.5.
		\end{itemize}
	
	\subsection{Alcance}
	El alcance de este documento comprende el levantamiento inicial de las necesidades de los usuarios y stakeholders, así como las características y funcionalidades más resaltantes del Sistema GNU /Ósmosis 2.0.
	
\section{Posicionamiento}
	
	\subsection{Enunciado del Problema}
	En la actualidad, el sistema GNU/Ósmosis 1.5, está basado en Dokeos (que a su vez está basado en Claroline) y presenta un código mal documentado, poco estructurado y de difícil mantenimiento. Adicionalmente, la comunidad de usuarios del sistema de Aulas Virtuales dentro de la Universidad Simón Bolívar ha solicitado nuevas funcionalidades, requerimientos que, poco a poco, han sido respondidos pero la complejidad del código ha dificultado y ralentizado el desarrollo de nuevas funcionalidades.
	
	\subsection{Oportunidad del Negocio}
	A pesar de que en el mercado ya existen sistemas de gestion y manejo de la educación, muchos de los sistemas de fuente abierta disponibles no poseen una base de código bien estructurada que ofrezca una mejor mantenibilidad y facilidad para el desarrollo e incorporación de nuevas funcionalidades.

\section{Stakeholders y Descripción de Usuarios}
	\subsection{Resumen del Stakeholder}
	En la tabla que se muestra a continuación, se ofrece un breve resumen de los stakeholders que participan en el desarrollo de este proyecto y su participación dentro del proyecto.

	\begin{center}
		\begin{tabular}{| p{0.15\textwidth} | p{0.45\textwidth} | p{0.4\textwidth} |}
			\hline  % Print horizontal line
			Nombre & Descripción & Responsabilidades \\ \hline
% 			
			DSM &
			Moderador técnico. Es la unidad encargada de
			dar apoyo, servicio y asesoría en el uso de recursos
			multimedia para programas de docencia,
			investigación y extensión de la USB. & 
			\begin{itemize}
				\item Garantizar el mantenimiento del sistema
				\item Hacer seguimiento al desarrollo del sistema
				\item Garantizar la disponibilidad de los recursos necesarios para el funcionamiento del sistema
			\end{itemize}
			\\ \hline
% 			
			\raggedright Fidel Gil, \linebreak Hermes Rodríguez &
			Mentor y Asesor, respectivamente. Prestan asesoría en cuanto a la determinación de los requerimientos del sistema y de las características de su comportamiento. &
			\begin{itemize}
				\item Prestar asesoría en la determinación de los requerimientos y características del sistema.
				\item Monitoreo del progreso del sistema
			\end{itemize}
			\\ \hline
% 			
			\raggedright
			Ana Gabriela Díaz,
				\linebreak
			Joaquín Windmüller,
				\linebreak
			José Lorenzo Rodríguez, &
			Desarrollador. Encargado del diseño e implementación del Sistema. &
			\begin{itemize}
				\item Desarrollo de la aplicación.
				\item Aporte de ideas para mejorar la funcionalidad de la aplicación.
			\end{itemize}
			\\ \hline
		\end{tabular}
	\end{center}

	\subsection{Resumen del Usuario}
	En la tabla que se muestra a continuación, se ofrece un breve resumen de los usuarios que maneja el sistema.
	
	\begin{center}
		\begin{tabular}{| p{0.20\textwidth} | p{0.4\textwidth} | p{0.4\textwidth} |}
			\hline  % Print horizontal line			
			\raggedright
			Nombre & Descripción & Responsabilidades \\ \hline
 			Administrador &
			Usuario del sistema que posee todos los permisos. &
			\begin{itemize}
			\item Realizar el mantenimiento del sistema
			\item Incorporar nuevas funcionalidades al sistema, de ser necesario.
			\item Mejorar las herramientas disponibles.
			\end{itemize}
			\\ \hline	

			\raggedright
 			Profesor &
			Persona con habilidades pedagógicas encargada de impartir la educación. Usuario principal o clave del sistema que tiene la posibilidad de usarlo como herramienta para impartir un curso. &
			\begin{itemize}
			\item Crear, actualizar, modificar o gestionar la información pedagógica que se quiera utilizar como ayuda de la metodología seleccionada para impartir el curso.
			\item Hacer seguimiento de las actividades planificadas
			\item Evaluar el desempeño del estudiante.
			\end{itemize}
			\\ \hline	

			\raggedright
 			Facilitador &
			Ayudante del profesor que contribuye con el proceso de enseñanza. &
			\begin{itemize}
			\item Se le atribuyen responsabilidades similares a las del profesor.
			\end{itemize}
			\\ \hline	

			\raggedright
 			Estudiante &
			Persona que cursa estudios. &
			\begin{itemize}
			\item Contribuir en el desarrollo de las actividades del curso.
			\item Hacer uso de las herramientas del sistema propuestas para complementar el proceso de aprendizaje.
			\item Evaluar o dar feedbacks del sistema y sus características.
			\end{itemize}
			\\ \hline		
		
		\end{tabular}
	\end{center}

	\subsection{Perfil del Stakeholder}
	\subsubsection{Moderador t�cnico}
	\begin{center}
		\begin{tabular}{| p{0.25\textwidth} | p{0.75\textwidth}|}
		 \hline  % Print horizontal line			
		\raggedright
		\textbf{Representante} & Aquellos que brindan asesoría, aportan ideas o encaminan el curso del proyecto \\ \hline
		\textbf{Descripción} & El asesor determina los requerimientos del sistema o sus funcionalidades además de la tecnología a usar.\\ \hline
		\textbf{Tipo} & Avanzado \\ \hline
		\textbf{Responsabilidades} & Establecer los lineamientos técnicos sobre los cuales se desarrolla el sistema.\\ \hline
		\textbf{Criterio de éxito} & Recibir un sistema que cumpla con las normas establecidas para el desarrollo del mismo. 
		\\ \hline
		\textbf{Actividades} & Establecer cuáles herramientas se pueden usar para el desarrollo de la aplicación. \\ \hline
		\end{tabular}
	\end{center}

	\subsubsection{Asesor}
	\begin{center}
		\begin{tabular}{| p{0.25\textwidth} | p{0.75\textwidth}|}
		 \hline  % Print horizontal line			
		\raggedright
		\textbf{Representante} & Aquellos que brindan asesoría, aportan ideas o encaminan el curso del proyecto
		\\ \hline
		\textbf{Descripción} & El asesor determina los requerimientos del sistema o sus funcionalidades además de la tecnología a usar.\\ \hline
		\textbf{Tipo} & Avanzado \\ \hline
		\textbf{Responsabilidades} & Brinda asesoría en cuanto a los requerimientos que ha de tener el sistema así como la tecnología a usar.\\ \hline
		\textbf{Criterio de éxito} & Recibir un sistema que cumpla con las normas establecidas para el desarrollo del mismo. 
		\\ \hline
		\textbf{Actividades} & En la revisión de requerimientos y el uso de la tecnología. \\ \hline
		\textbf{Comentarios/Otros} & Problemas de comunicación entre el asesor y los desarrolladores del sistema. \\ \hline
		\end{tabular}
	\end{center}

	\subsubsection{Mentor}
	\begin{center}
		\begin{tabular}{| p{0.25\textwidth} | p{0.75\textwidth}|}
		 \hline  % Print horizontal line			
		\raggedright
		\textbf{Representante} & Aquellos que vigilan el desarrollo del proyecto, en cuanto a prácticas metodológicas.
		\\ \hline
		\textbf{Descripción} & El mentor vigila el uso de las prácticas metodológicas para el desarrollo de software además del avance del proyecto en general.\\ \hline
		\textbf{Tipo} & Avanzado \\ \hline
		\textbf{Responsabilidades} & Brinda asesoría en el uso del RUP. Vigila el desarrollo del proyecto.\\ \hline
		\textbf{Criterio de éxito} & Recibir un sistema que cumpla con las normas establecidas para el desarrollo del mismo. 
		\\ \hline
		\textbf{Actividades} & En la generación de requerimientos, y aporte de métodos, prácticas y experiencias en el desarrollo de software. \\ \hline
		\textbf{Comentarios/Otros} & Problemas de comunicación entre el mentor y los desarrolladores del sistema. \\ \hline
		\end{tabular}
	\end{center}

	\subsubsection{Desarrollador}
	\begin{center}
		\begin{tabular}{| p{0.25\textwidth} | p{0.75\textwidth}|}
		 \hline  % Print horizontal line			
		\raggedright
		\textbf{Representante} & Aquellos encargados del diseño y desarrollo de la aplicación.\\ \hline
		\textbf{Descripción} & El desarrollador diseña e implementa la aplicación para ser usada. Interviene en el mejoramiento o aporte de requerimientos.\\ \hline
		\textbf{Tipo} & Avanzado \\ \hline
		\textbf{Responsabilidades} & Desarrolla la aplicación y aporta ideas creativas para su implementación o diseño.
		\\ \hline
		\textbf{Criterio de éxito} & Desarrollar y completar un sistema que cumpla con las normas establecidas para el desarrollo del mismo. 
		\\ \hline
		\textbf{Actividades} & En el desarrollo de la aplicación en general
		\\ \hline
		\textbf{Comentarios/Otros} & Dificultad en el aprendizaje del uso de las herramientas necesarias para desarrollar el sistema.
		Tiempo limitado y dedicaci�n no exclusiva al desarrollo de la aplicación. \\ \hline
		\end{tabular}
	\end{center}

	\subsection{Perfil del Usuario}

	\subsubsection{Administrador}
	\begin{center}
		\begin{tabular}{| p{0.25\textwidth} | p{0.75\textwidth}|}
		 \hline  % Print horizontal line			
		\raggedright
		\textbf{Representante} & 
		\\ \hline
		\textbf{Descripción} & \\ \hline
		\textbf{Tipo} &  \\ \hline
		\textbf{Responsabilidades} & \\ \hline
		\textbf{Criterio de éxito} & \\ \hline
		\textbf{Actividades} &  \\ \hline
		\textbf{Comentarios/Otros} &  \\ \hline
		\end{tabular}
	\end{center}

	\subsubsection{Profesor}
	\begin{center}
		\begin{tabular}{| p{0.25\textwidth} | p{0.75\textwidth}|}
		 \hline  % Print horizontal line			
		\raggedright
		\textbf{Representante} & 
		\\ \hline
		\textbf{Descripción} & \\ \hline
		\textbf{Tipo} &  \\ \hline
		\textbf{Responsabilidades} & \\ \hline
		\textbf{Criterio de éxito} & \\ \hline
		\textbf{Actividades} & \\ \hline
		\textbf{Comentarios/Otros} & \\ \hline
		\end{tabular}
	\end{center}

	\subsubsection{Facilitador}
	\begin{center}
		\begin{tabular}{| p{0.25\textwidth} | p{0.75\textwidth}|}
		 \hline  % Print horizontal line			
		\raggedright
		\textbf{Representante} & 
		\\ \hline
		\textbf{Descripción} & \\ \hline
		\textbf{Tipo} & \\ \hline
		\textbf{Responsabilidades} & \\ \hline
		\textbf{Criterio de éxito} & \\ \hline
		\textbf{Actividades} & \\ \hline
		\textbf{Comentarios/Otros} & \\ \hline
		\end{tabular}
	\end{center}

	\subsubsection{Estudiante}
	\begin{center}
		\begin{tabular}{| p{0.25\textwidth} | p{0.75\textwidth}|}
		 \hline  % Print horizontal line			
		\raggedright
		\textbf{Representante} & \\ \hline
		\textbf{Descripción} & \\ \hline
		\textbf{Tipo} &  \\ \hline
		\textbf{Responsabilidades} & \\ \hline
		\textbf{Criterio de éxito} & \\ \hline
		\textbf{Actividades} & \\ \hline
		\textbf{Comentarios/Otros} & \\ \hline
		\end{tabular}
	\end{center}

	\subsection{Necesidades de los Stakeholders o Usuarios}
	\begin{center}
		\begin{tabular}{| p{0.2\textwidth} | p{0.12\textwidth} | p{0.25\textwidth} | p{0.3\textwidth} | p{0.3\textwidth}|}
		\hline  % Print horizontal line			
			\raggedright
			Necesidad & Prioridad & Personas \linebreak Relacionadas & Solución Actual & Soluci�n \linebreak Propuesta \\ \hline
			
		
		\end{tabular}
	\end{center}
	
	\subsection{Alternativas y Competencia}
	Actualmente existe una gran variedad de sistemas de gesti�n de aprendizaje en el mercado, a continuación se listan algunos de ellos:
	\begin{itemize}
	\item Moodle 1.8
	\item Dokeos 1.5.4
	\item Sakai 2.3
	\item Claroline 1.8.1
	\item ATutor 1.5.4
	\item Blackboard Learning System Vista 4.1 Enterprise License
	\item Desire2Learn 8.2
	\end{itemize}

	Ver anexo Sistemas de Gestión de Aprendizaje para detalles de las características de cada sistema y comparación entre ellos 

\section{Descripción del Sistema}

\subsection{Perspectiva del Sistema}

\subsection{Resumen de Capacidades}

\subsection{Licencias e Instalación.}

\section{Características del Sistema}

Algunas de las herramientas o funcionalidades que brinda el sistema son:

\begin{itemize}
\item Foros
\item Intercambio de Archivos
\item Mensajería Interna
\item Blog
\item Wiki
\item Chat
\item Pizarra
\item Lecciones
\item Enlaces
\item Calendario/Agenda
\item Capacidades de búsqueda
\item Trabajo desconectado
\item Agregador de Noticias
\item Grupos de Trabajo
\item Ayuda y Orientación del curso
\item Ayuda del Sistema
\item Comunidad
\item Portafolios
\item Administración automatizada de pruebas
\item Control de calificaciones del curso 
\item Manejo del curso
\item Seguimiento de usuarios
\item Posibilidad de compartir/reusar información
\item Plantillas del curso.
\item Herramientas de diseño de cursos.
\end{itemize}

Para ver información completa de las características antes mencionadas y justificación de las mismas ver anexo 

\section{Restricciones}

\section{Rangos de Calidad}

\section{Precedencia y Prioridad}

\section{Otros Requerimientos de producto}

\subsection{Estándares de Aplicación}
El sistema debe ser desarrollado bajo el estándar actual de XHTML  y CSS , en un ambiente cliente-servidor, y desarrollado en PHP5.

\subsection{Requerimientos del sistema}

\subsection{Requerimientos de desempeño }

\section{Requerimientos de documentación}

\section{Riesgos}

\begin{itemize}
 \item Dedicación no exclusiva al desarrollo del sistema
\item Estado de salud de los miembros del equipo
\item Disponibilidad del tutor o caulquier otra persona que esté involucrada en la dirección del proyecto de desarrollo.
\item Fallas en la plataforma tecnológica utilizada (servidores, computadores, acceso a internet)
\end{itemize}


\subsection{Ayuda en línea}

\subsection{Manual del usuario}

%Create the environment for the bibliography.  Since there is only one
%reference, set the label width to be one character (I shall follow
%convention as use the number '9'.  This is because it helps to remind
%that it is the maximum number of refs that is now permitted by that
%width).
\begin{thebibliography}{9}
%The \bibitem is to start a new reference.  Ensure that the cite_key is
%unique.  You don't need to put each element on a new line, but I did
%simply for readability.
	\bibitem{lamport94}
	  Leslie Lamport,
	  \emph{\LaTeX: A Document Preparation System}.
	  Addison Wesley, Massachusetts,
	  2nd Edition,
	  1994.

\end{thebibliography} %Must end the environment

\end{document}  %End of document.
